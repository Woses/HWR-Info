\clearpage
\section*{Begriffs-Legende}

Damit der Einstieg leichter fällt: Hier sind typische Begriffe aus Schule, Alltag und Hochschule im Vergleich.  
Links steht das, was du vielleicht aus der Schule kennst, rechts die Begriffe, die an der Uni benutzt werden.

\vspace{5mm}

\begin{center}
\resizebox{\textwidth}{!}{%
\begin{tabular}{|p{0.35\textwidth}|p{0.55\textwidth}|}
\hline
\textbf{Früher (Schule / Alltag)} & \textbf{Heute (Hochschule)} \\ \hline
Lehrer*innen & Dozent*innen / Professor*innen \\ \hline
Direktor & Dekan \\ \hline
Unterrichtsstunde & Vorlesung / Seminar \\ \hline
Hausaufgaben & Selbststudium / Übungsaufgaben \\ \hline
Klassenarbeit & Klausur / Prüfung \\ \hline
Schuljahr & Semester \\ \hline
Klassenbuch & Modulhandbuch / Prüfungsordnung \\ \hline
Zeugnis & Transcript of Records (Notenübersicht) \\ \hline
Schulpflicht & Rückmeldepflicht (Semesterbeitrag zahlen) \\ \hline
Klassenkamerad*innen & Kommiliton*innen \\ \hline
Pausenhof & Campus / Mensa / Cafeteria \\ \hline
Pausenklingel & Vorlesungszeiten (c.t. / s.t.) \\ \hline
Klassenlehrer & Studiengangsleiter / Mentor \\ \hline
Elternabend & Fachschaftssitzung / AStA-Vollversammlung \\ \hline
Nachsitzen & Zweittermin / Wiederholungsprüfung \\ \hline
Schulweg & Pendeln zur Hochschule \\ \hline
Klassensprecher & Fachschaftsvertreter / StuPa-Abgeordnete \\ \hline
Schülerausweis & Studierendenausweis \\ \hline
Schülervertretung (SV) & Studierendenvertretung (AStA, StuPa, Fachschaft) \\ \hline
Pausenbrot & Mensaessen \\ \hline
Bleistift & Laptop / Tablet \\ \hline
Tafel & Beamer / Whiteboard \\ \hline
Arbeitsblätter & Skripte / Folien \\ \hline
Schulheft & Mitschrift / Notiz-App \\ \hline
Nachhilfe & Tutorium \\ \hline
AG (Arbeitsgemeinschaft) & Hochschulgruppe / Verein \\ \hline
Schulausflug & Exkursion \\ \hline
Sportfest & Hochschulsport / Campuslauf \\ \hline
Schulfest & Campusfest \\ \hline
Bibliothek in der Schule & Hochschulbibliothek \\ \hline
Abschlussparty & Ersti-Party / Semesterparty \\ \hline
Ferien & Semesterferien / vorlesungsfreie Zeit \\ \hline
Zeugniskonferenz & Prüfungsausschuss \\ \hline
Klassenzimmer & Hörsaal / Seminarraum \\ \hline
Sitzordnung & freie Platzwahl \\ \hline
Schultasche & Laptop-Rucksack \\ \hline
\end{tabular}%
}
\end{center}
\begin{center}
\resizebox{\textwidth}{!}{%
\begin{tabular}{|p{0.35\textwidth}|p{0.55\textwidth}|}
\hline
Sporthalle & Hochschulsportzentrum \\ \hline
Hausmeister & Facility Management / Technischer Dienst \\ \hline
Sekretariat & Prüfungsservice / Fachbereichssekretariat \\ \hline
Stundenplan (vorgegeben) & Stundenplan (selbst gebaut) z.B. auf der Rückseite dieses Heftes \\ \hline
Pausenzeiten & Blockzeiten (90 Min. Vorlesung) \emph{*ba-dum-tssz*} \\ \hline
Schulhoftratsch & Campusleben / Discord / WhatsApp-Gruppen \\ \hline
Arbeitsgruppen in der Schule & Projektgruppen an der Uni \\ \hline
Klassentest & Kurztest / Übungsabgabe \\ \hline
Leistungsnachweis & Credit Points (ECTS) \\ \hline
Klassenausflug & Exkursion / Studienfahrt \\ \hline
Schulmappe & Modulübersicht / Studienplan \\ \hline
Schüler-ID & Matrikelnummer \\ \hline
Klassenbuch-Eintrag & Notenverbuchung im System (Prüfungsamt) \\ \hline
Nachhilfelehrer*in & Tutor*in / Übungsleiter*in \\ \hline
Schulsprecher*in & AStA-Vorsitz / Fachschaftsvorsitz \\ \hline
Schulsekretär*in & Fachbereichssekretariat / Prüfungsservice \\ \hline
Stundenplanänderung & Vorlesungsausfall / Ersatztermin per E-Mail \\ \hline
Zeugnisausgabe & Notenbekanntgabe im Portal \\ \hline
Schulwechsel & Hochschulwechsel / Studiengangswechsel \\ \hline
Sitzenbleiben & Studienverlängerung / Semester wiederholen \\ \hline
Schülerakte & Studierendenakte im Prüfungsamt \\ \hline
Elternunterschrift & Eigenverantwortung (alles selbst regeln) \\ \hline
Taschengeld & BAföG / Nebenjob / Wohngeld \\ \hline
Vorlesungsverzeichnis & LSF / HISinOne / Online-Portal \\ \hline
Klassentagebuch & Moodle / ILIAS / LEA (Lernplattform) \\ \hline
Schulordnung & Studienordnung / Prüfungsordnung \\ \hline
Klassenfahrt & Auslandssemester / Erasmus \\ \hline
Vertretungsplan & Raumänderungen im Portal / E-Mail \\ \hline
Klassenbuchführung durch Lehrer & Anwesenheitsliste (manchmal digital) \\ \hline
Abschlussexkursion & Exkursion / Studienfahrt \\ \hline
Schulheftabgabe & Übungsblattabgabe / Online-Upload \\ \hline

Projektwoche & Blockseminar / Hackathon / Projektmodul \\ \hline
Schulsozialarbeit & Zentrale Studienberatung / Psychologische Beratung \\ \hline
Arbeitsgemeinschaft „Schülerzeitung“ & Ersti-Heft/ Hochschulmagazin \\ \hline
Klassensprecherwahl & Fachschaftswahlen / Hochschulwahlen \\ \hline
Elternsprechtag & Sprechstunde bei Dozent*innen \\ \hline
Krankmeldung bei Eltern & Attest an Prüfungsamt / Studienbüro \\ \hline

\end{tabular}%
}
\end{center}


\begin{center}
\emph{\textbf{Merke:} Schule ist Pflicht, Studium ist Freiheit und Verantwortung – aber mit den richtigen Begriffen geht der Start leichter.}
\end{center}



\clearpage

\begin{center}
\resizebox{\textwidth}{!}{%
\begin{tabular}{|p{0.35\textwidth}|p{0.55\textwidth}|}
\hline
\textbf{Anlaufstelle} & \textbf{Wofür?} \\ \hline
\textbf{Fachbereichssekretariat} & Erste Anlaufstelle für Organisatorisches, Kontakt zum Dekanat, Unterstützung bei Formularen \\ \hline
\textbf{Prüfungsservice} & Anmeldung und Rücktritt von Prüfungen, Zeugnisse, Bescheinigungen \\ \hline
\textbf{Prüfungsausschuss} & Sonderfälle, Fristverlängerungen, Nachteilsausgleich, Streitfälle \\ \hline
\textbf{Fachschaft} & Hilfe von Studierenden für Studierende, Klausurensammlung, Events, Beratung \\ \hline
\textbf{AStA} (Allgemeiner Studierendenausschuss) & Studierendenvertretung, Beratung (Recht, Soziales, Finanzen), Kulturangebote \\ \hline
\textbf{StuPa} (Studierendenparlament) & Hochschulpolitik, Verabschiedung des AStA-Haushalts \\ \hline
\textbf{Dekanat} & Leitung des Fachbereichs, strategische Entscheidungen, wichtige Ansprechpartner*innen \\ \hline
\textbf{Studierendenwerk Bonn} & BAföG, Wohnen, psychologische Beratung, Kindertagesstätten, Mensa \\ \hline
\textbf{PBS} (Psychologische Beratungsstelle) & Hilfe bei Stress, Krisen, psychischen Belastungen \\ \hline
\textbf{Sozialberatung Studierendenwerk} & Wohngeld, BAföG-Fragen, finanzielle Notlagen, Stipendienberatung \\ \hline
\textbf{Bibliothek (H-BRS)} & Literaturrecherche, Ausleihe, Lernplätze, E-Ressourcen, Citavi, Fernleihe \\ \hline
\textbf{IT-Services} & WLAN, VPN, Campus-Accounts, E-Mail, Softwarelizenzen (Office, JetBrains, etc.) \\ \hline
\textbf{Mensa / Cafeteria} & Verpflegung am Campus, günstiges Essen, Treffpunkt für Studierende \\ \hline
\textbf{Hochschulsport} & Sportkurse, Fitnessangebote, Hochschulmeisterschaften \\ \hline
\textbf{Career Service} & Bewerbungstraining, Praktikumsberatung, Jobbörse, Workshops \\ \hline
\textbf{International Office} & Auslandssemester, Erasmus, Beratung für internationale Studierende \\ \hline
\textbf{Sprachenzentrum} & Sprachkurse, Schreibzentrum, Sprechwerkstatt, Zertifikate \\ \hline
\textbf{Studierendenservice} & Rückmeldung, Beurlaubung, Studienbescheinigung, Studentenausweis \\ \hline
\textbf{Vertrauenspersonen} & Anlaufstelle bei Diskriminierung, Konflikten, Problemen \\ \hline
\textbf{Wohnheimtutor*innen} & Ansprechpartner im Wohnheim, Events, Gemeinschaftsleben \\ \hline
\end{tabular}%
}
\end{center}


\begin{center}
\emph{\textbf{Tipp:} Viele dieser Stellen erreichst du per Mail oder findest ihre Infos auf der H-BRS Webseite oder heir im Heft. Die Fachschaft hilft dir, wenn du nicht weißt, wohin du musst.}
\end{center}


\clearpage
\section*{Abkürzungs-Legende}

Hier sind typische Abkürzungen, die dir im Studium an der H-BRS begegnen können.  

\vspace{5mm}

\begin{center}
\resizebox{\textwidth}{!}{%
\begin{tabular}{|p{0.25\textwidth}|p{0.65\textwidth}|}
\hline
\textbf{Abkürzung} & \textbf{Bedeutung} \\ \hline
VL /V & Vorlesung \\ \hline
Üb / Ü / Tut & Übung / Tutorium \\ \hline
HS & Hörsaal \\ \hline
Sem & Seminar \\ \hline
Bib & Bibliothek \\ \hline
Klausur & Schriftliche Prüfung \\ \hline
SP & Servicepoint (z. B. FB02 Servicepoint) \\ \hline
PO & Prüfungsordnung \\ \hline
FSR & Fachschaftsrat \\ \hline
Fachschaft & Vertretung der Studierenden eines Fachbereichs \\ \hline
WiSe & Wintersemester \\ \hline
SoSe & Sommersemester \\ \hline
PBS & Psychologische Beratungsstelle (Studierendenwerk Bonn) \\ \hline
PA & Prüfungsausschuss \\ \hline
SSC & Student Service Center \\ \hline
AAA & Akademisches Auslandsamt (International Office) \\ \hline
SHK & Studentische Hilfskraft \\ \hline
WHK & Wissenschaftliche Hilfskraft \\ \hline
WiMi & Wissenschaftlicher Mitarbeiter \\ \hline
Hiwi & Synonym für SHK / WHK \\ \hline
ZBR & Zentrale Bibliothek Rheinbach \\ \hline
SARBS & Service- und Antragsportal der H-BRS (Raumbuchung etc.) \\ \hline
StuPa & Studierendenparlament (Legislativorgan der Studierendenschaft) \\ \hline
AStA & Allgemeiner Studierendenausschuss (Exekutivorgan der Studierendenschaft) \\ \hline
FB02 & Fachbereich Informatik (Numerierung an der H-BRS) \\ \hline
PDF & Portable Document Format \\ \hline
CSS & Cascading Style Sheets \\ \hline
JS & JavaScript \\ \hline
IP & Internet Protocol \\ \hline
LAN & Local Area Network \\ \hline
MAC & Media Access Control (Adresse der Netzwerkkarte) \\ \hline
IMAP & Internet Message Access Protocol (für E-Mails) \\ \hline
SMTP & Simple Mail Transfer Protocol (zum E-Mail-Versand) \\ \hline
HTML & HyperText Markup Language \\ \hline
\end{tabular}%
}
\end{center}
\begin{center}
\resizebox{\textwidth}{!}{%
\begin{tabular}{|p{0.25\textwidth}|p{0.65\textwidth}|}
\hline
\textbf{Abkürzung} & \textbf{Bedeutung} \\ \hline
LP / CP & Leistungspunkte / Credit Points \\ \hline
ECTS & European Credit Transfer System (Leistungspunkte) \\ \hline
MHB & Modulhandbuch \\ \hline
VPN & Virtual Private Network (für internen Hochschulzugang) \\ \hline
LEA & Lernplattform ILIAS der H-BRS \\ \hline
ILIAS & Integriertes Lern-, Informations- und Arbeitskooperations-System \\ \hline
ZSB & Zentrale Studienberatung \\ \hline
DFN & Deutsches Forschungsnetz (VPN-Provider) \\ \hline
Moodle & Alternative Lernplattform (teilweise in Kursen genutzt) \\ \hline
HRZ & Hochschulrechenzentrum (Uni Bonn und Service) \\ \hline
ZKI & Zentrum für Kommunikation und Informationstechnik \\ \hline
PBS-Online & Online-Terminportal der Beratungsstelle \\ \hline
Deutschland- semesterticket & Semesterticket für ganz Deutschland \\ \hline
VRR / VRS & Verkehrsverbund Rhein-Ruhr / Rhein-Sieg \\ \hline
SWB(-V) & Stadtwerke Bonn (Verkehr)(Für Bus und Bahn) \\ \hline
KVB & Kölner Verkehrsbetriebe \\ \hline
ZWEK & Zentrum für Wissenschaftliche Weiterbildung \\ \hline
c.t. & cum tempore = 15 Minuten später (akademisches Viertel) \\ \hline
s.t. & sine tempore = pünktlich zur angegebenen Zeit \\ \hline
BAföG & Bundesausbildungsförderungsgesetz (Studienfinanzierung) \\ \hline
HISinOne & Neues Hochschul-Informations-System (Studierendenverwaltung)(APOLLO) \\ \hline
\end{tabular}%
}
\end{center}

\fancyfoot[C]{%
	\small
	\begin{tikzpicture}[remember picture,overlay]
		%\druckrand
	\end{tikzpicture}
	\color{fscolor}
	\flushbottom
	\ifcase\value{page}%
	% empty for page 0
        \or Warum benutzt ein Hacker immer Sonnenbrillen? -- Weil er sich vor Phishing-Attacken schützen will.
	\or Wie nennt man ein ostdeutsches Computermodell? -- Thüring-Maschine!
	\or ($\mathbb{Z}$,$+$) und ($\mathbb{Q}\setminus\{0\}$,$\cdot$) gehen ins Kino. $\mathtt{Sym}(3)$ hat eigentlich keine Lust, geht aber trotzdem mit. Gruppenzwang.
	\or Vier singende Informatiker bilden einen … Quad-Chor.
    \or Warum sind Wirtschaftsinformatiker keine guten Gärtner? -- Weil sie immer den Workflow automatisieren wollen, aber Pflanzen nicht mitmachen.
	\or Die meistgestellten Fragen: Ingenieur: Wie geht das? Ökonom: Wie teuer wird das? Mathematiker: Wie kann man das verbessern? Physiker: Möchten Sie dazu Ketchup?
    \or Ein Roboter geht in ein Restaurant und bestellt: \enquote{Bitte laden Sie mich mit dem Tagesmenü neu auf.}
	\or Warum sind Hacker so gute Musiker? -- Weil sie immer auf den richtigen Key achten!
    \or Cyber-Security-Bewerbungsgespräch: \enquote{Haben Sie Erfahrung mit Firewalls?} -- \enquote{Ja, ich habe sie gebaut und angezündet!}
	\or Woran erkennt man, dass eine Zahnärztin früher einmal Mathematikerin war? Das Einzige, was sie tut, ist Wurzelziehen.
    \or Ihr Computer ist zu langsam? -- Schon mal versucht, ihm einen Kaffee anzubieten?
	\or Eine Ingenieurin, ein theoretischer und ein Experimentalphysiker wachen nachts auf und merken, dass ihre Häuser brennen. Was tun sie? Die Ingenieurin rennt zum Feuerlöscher, löscht damit den Brand und legt sich wieder schlafen. Der theoretische Physiker setzt sich an den Schreibtisch, rechnet, nimmt dann ein Glas Wasser und schüttet es so auf das Feuer, dass es erlischt. Der Experimentalphysiker verbrennt auf der Suche nach einem Thermometer.
    \or Warum klebt auf allen Intel-Rechnern \enquote{Intel inside}? Ein Warnhinweis ist einfach nötig.
    \or Warum kann ein Wirtschaftsinformatiker nicht ohne Internetverbindung arbeiten? -- Weil er sonst keine KPI-Dashboard-Updates bekommt.
    \or Es gibt 10 Arten von Menschen. Die, die Binärcode verstehen und die, die ihn nicht verstehen.
    \or Warum sind Wirtschaftsinformatiker so gut in Verhandlungen? -- Weil sie wissen, wie man Kosten und Risiken minimiert, während der Gewinn maximiert wird.
    \or Ein Wirtschaftsinformatiker sagt zu seinem Team: \enquote{Wir müssen den Workflow optimieren!} -- Team: \enquote{Heißt das, weniger Meetings?} -- \enquote{Nein, effizientere Meetings!}
	\or Was ist die Lieblingsbeschäftigung von Bits? Bus fahren!
    \or Warum trinken Informatiker keinen Kaffee? Weil sie bei der Fehlerbehebung nicht in eine Endlosschleife geraten wollen.
	\or Der Computer rechnet mit allem -- nur nicht mit seinem Besitzer.
    \or Warum sind Informatiker schlecht im Kochen? -- Weil sie zu viel Zeit damit verbringen, den Rezeptcode zu debuggen.
    \or I keep a fork in my PC, incase I wanted to have a byte.
	% Seite 50
    \or Warum haben Roboter keine Angst vor der Zukunft? -- Weil sie ihre Algorithmen bereits optimiert haben.
	%\or Wie oft kann man 7 von 83 abziehen, und was bleibt am Ende übrig? Man kann so oft wie man will 7 von 83 abziehen, und es bleibt jedes Mal 76 über.
	\or Was sagt ein Mathematiker zu seiner Frau, nachdem er sie im Bett so richtig scharf gemacht hat? \enquote{Der Rest ist trivial, den kannst du dir als Übungsaufgabe selbst herleiten.}
	\or Die Mathelehrerin sagt: \enquote{Die Klasse ist so schlecht in Mathe, dass sicher 90\% dieses Jahr durchfallen werden.} Ein Schüler im Hintergrund: \enquote{Aber so viele sind wir doch gar nicht!}
	\or Fragt ein Mathematiker den anderen: \enquote{Ey, wie hoch ist diese Schranke?} Der andere klettert rauf, misst, kommt runter und sagt: \enquote{4,32 Meter.} Sagt der Erste: \enquote{Bist du doof! Warum hast du nicht gewartet, bis die Schranke runter kommt?} Sagt der andere: \enquote{Nee, du bist doof, ich wollte ja wissen wie hoch sie ist, nicht wie breit!}
    \or Ein Informatiker ist jemand, der die Lösung eines Problems versteht, aber nicht das Problem.
    \or Ein Informatiker und ein Programmierer gehen in ein Café. Der Programmierer sagt: \enquote{Ich nehme einen Kaffee.} Der Informatiker antwortet: \enquote{Ja, aber was macht der Kaffee in meinem Stack?}
    \or Ein Informatiker verliert nie seine Arbeit. Er hat ein Backup. Zwei, um genau zu sein. Ok, drei, verteilt auf vier Festplatten. Eine davon ist in der Cloud.
	\or Ein Statistiker kann seinen Kopf in den Backofen und seine Füße in Eiswasser stecken, und er wird sagen: \enquote{Im Durchschnitt geht es mir gut.}
    \or Haben Sie den Computer schon in den Ruhemodus geschickt? Vielleicht braucht er einfach mal eine Pause.
    \or Was ist die Lieblingsaufgabe einer Turingmaschine? -- Das Halteproblem zu vermeiden.
	\or Was fährt auf Schienen und kehrt die Exponentialfunktion um? -- Eine Dampf-log.
	% Seite 80
    \or Warum lieben Wirtschaftsinformatiker Tabellen? -- Weil sie immer eine Formel für jedes Problem haben.
	\or Wenn der Computer abstürzt, einfach abwarten – er lernt nur, wie es geht.
    \or Warum können Informatiker keine Zaubertricks? -- Weil sie immer den Code dahinter sehen wollen.
    \or Ein Wirtschaftsinformatiker zur IT: \enquote{Können Sie mir bitte diese Datei senden?} -- IT: \enquote{Schick sie dir per Mail?} -- \enquote{Nein, besser als Excel-Tabelle!}
	\or Wusstest du, dass fast alle Menschen mehr Beine haben als der Durchschnitt?
    \or Ein Informatiker baut einen Roboter. Der Roboter sagt: \enquote{Ich denke, also kompiliere ich.}
	\or Ein theoretischer Physiker im Zug fragt die Schaffnerin: \enquote{Entschuldigung, hält an diesem Zug auch Genf?}
    \or Warum sind Turingmaschinen so wortkarg? -- Weil sie nur Zustandsübergänge beschreiben können.
    \or Die beste Lösung für IT-Probleme? Einfach den PC ein Wochenende in Urlaub schicken!
	\or Eine Mathematikerin will ihren neuesten Beweis als Bild aufhängen. Sie nimmt Nagel und Hammer und hält den Nagel mit dem Kopf zur Wand. Gerade als sie zuschlagen will, schaut sie noch mal genau hin -- und stutzt. Nach fünf Minuten konzentrierten Hinschauens und Überlegens hat sie's: \enquote{Das ist ein Nagel für die gegenüberliegende Wand!}
	\or Sitzt ein Mathematiker in der Kneipe und saugt am Rand seines Glases. Da kommt ein zweiter Mathematiker vorbei und fragt, warum er denn nicht wie alle anderen trinke. Darauf der erste: \enquote{Nach dem Satz von Gauß muss das auch so klappen.}
    \or Warum machen Hacker keine Fehler? -- Weil sie immer in den Logs nachsehen können, was schiefgelaufen ist.
	\or Abiturprüfung. Schulleiter zum Abiturienten: \enquote{Kennen wir uns nicht?} Abiturient: \enquote{Ja, vom Mathe-Abi im letzten Jahr.} Schulleiter: \enquote{Ach so, ja. Aber heute wird's schon klappen. Wie lautete denn damals die erste Frage, die ich Ihnen gestellt habe?} Abiturient: \enquote{Kennen wir uns nicht…}	
    \or Wenn die Festplatte voll ist, einfach das Fenster öffnen und lüften!
    \or Was ist der Unterschied zwischen einem Ökonom und einem Meteorologen? -- Der Meteorologe kann das Wetter nicht vorhersagen, aber der Ökonom erklärt im Nachhinein, warum es anders gekommen ist.
	\or Wie bringen Mathematiker*innen ihre Gegner um, ohne eine Mordwaffe zu hinterlassen? Sie legen ihnen einen Kreis um den Hals und lassen den Radius gegen null gehen.
    \or Warum nehmen Informatiker immer eine Taschenlampe mit in den Serverraum? -- Damit sie die Bits in der Dunkelheit sehen können.
	% Seite 70	
	\or Was haben eine Mathematikerin und ein Physiker gemeinsam? Beide sind dumm -- mit Ausnahme der Mathematikerin.
	\or Das Passwort ist falsch? Versuchen Sie es rückwärts, vielleicht war der Caps Lock an!
	\or Werner Heisenberg wird auf der Autobahn von der Polizei angehalten. Die Beamtin verlangt nach Führer- und Fahrzeugschein, schaut sich diese an und fragt: \enquote{Herr Heisenberg, wissen Sie, wie schnell Sie gefahren sind?} \enquote{Nein}, antwortet Heisenberg, \enquote{aber ich weiß, wo ich jetzt bin!}
	\or Was sind 10 Physiker in Salzsäure? Ein gelöstes Problem!
    \or Warum setzen sich Informatiker gerne in den Schatten? -- Weil sie keine unnötigen Threads starten wollen.
    \or Was ist das Lieblingsgetränk eines Roboters? -- Lötzinn und Öl.
	\or Haben Sie es mal mit \enquote{F5} probiert? -- Aktualisieren kann Wunder bewirken!
    \or Ein Statistiker läuft im Wald und sieht einen Bären. Er bleibt ruhig und sagt: \enquote{Keine Sorge, im Mittel greifen Bären Menschen nicht an.}
    \or Warum gehen Cyber-Security-Experten selten ins Casino? -- Weil sie wissen, dass das Haus immer gewinnt.
    \or Ein Informatiker zur Turingmaschine: \enquote{Kannst du das Problem lösen?} -- Turingmaschine: \enquote{Das hängt von meinem Band ab.}
	\or Treffen sich zwei Funktionen in der Unendlichkeit. Sagt die eine: \enquote{Ich differenzier dich gleich!} Sagt darauf die andere: \enquote{Ätsch, ich bin die e-Funktion.}
    \or Was passiert, wenn ein Wirtschaftsinformatiker auf eine IT-Störung stößt? -- Er erstellt zuerst eine SWOT-Analyse, bevor er den Stecker zieht.
    \or Wie wünschen sich Programmierer Frohe Weihnachten? \enquote{Merry Christmas() and a Happy New Year();}
	\or Treffen sich zwei Geraden. Sagt die eine: \enquote{Beim nächsten Mal gibst du einen aus.}
	\or Ein Gedicht. $\mathbb{Z}$ ist fromm. $\mathbb{Q}$ ist es nicht. Denn $\mathbb{Q}$ ist dicht.
	\or Auf der Heizung liegt ein Ziegelstein. Prüferin: \enquote{Warum ist der Stein auf der Heizung abgewandten Seite wärmer?} Prüfling: \enquote{Äh, vielleicht wegen Wärmeleitung und so?} Prüferin: \enquote{Nein, weil ich ihn gerade umgedreht habe.}
    \or Warum gehen Informatiker nicht gerne nach draußen? Sie können die Sonne nicht minimieren.
    \or Warum sind Wirtschaftler so gut im Verhandeln? -- Weil sie gelernt haben, dass es nicht ums Gewinnen, sondern ums Verhandeln geht.
    \or Die meistgestellten Fragen: Ingenieur: Wie geht das? Ökonom: Wie teuer wird das? Mathematiker: Wie kann man das verbessern? Physiker: Möchten Sie dazu Ketchup?
	% Seite 40
    \or Ein Cyber-Security-Experte betritt eine Bar: \enquote{Ich nehme ein Root-Bier, aber ohne Berechtigungen.}
	\or Vorlesung in Betriebswirtschaft: \enquote{Was sind die drei wichtigsten Wörter für den Erfolg im Management?} -- \enquote{Es kommt darauf an.}
    \or Noch nie wurde ein IT-Leiter gefeuert, weil er Produkte von IBM, Microsoft, HP etc. eingekauft hat (leider kein Witz).
	\or Warum sind Birnen auch Homomorphismen? Sie haben Kerne.
    \or Ein Informatiker bekommt eine Fehlermeldung und fragt: \enquote{Hat es mal jemand mit einem Zaubertrank probiert?}
    \or Warum redet mein Drucker nicht mehr mit mir? -- Wahrscheinlich fühlt er sich nicht mehr verbunden.
    \or Was ist die Lieblingsdiät eines Informatikers? -- Die If-Diät: Wenn Hunger == wahr, dann essen.
	\or Treffen sich zwei Matrizen. Sagt die eine: \enquote{Komm wir gehen in den Wald und machen A hoch minus 1.} Sagt die andere: \enquote{Mensch, bist Du invers!}
	\or Wieso sind Hausdorff-Räume unsolidarisch? -- Jeder ist sich selbst der Nächste.
    \or Warum sind Turingmaschinen so schlecht in Smalltalk? -- Weil sie nur mit endlosen Bändern arbeiten.
    \or Schon versucht, das Internet auszuschalten und wieder anzuschalten? Vielleicht hilft’s ja!
	\or Eine Ingenieurin denkt, dass Gleichungen eine Annäherung an die Realität sind. Ein Physiker denkt, dass die Realität eine Annäherung an die Gleichungen ist. Einem Mathematiker ist es egal.
    \or Was ist das Lieblingsspiel eines Wirtschaftsinformatikers? -- Scrum Monopoly: Man muss das Haus agilisieren, bevor es pleite geht.
	\or Ingenieurin zum Mathematiker: \enquote{Ich finde Ihre Arbeit ziemlich monoton.} Mathematiker: \enquote{Mag sein, dafür ist sie aber stetig und nicht beschränkt.}
	\or Behauptung: Eine Katze hat neun Schwänze. Beweis: Keine Katze hat acht Schwänze. Eine Katze hat einen Schwanz mehr als keine Katze. Deshalb hat eine Katze neun Schwänze.
	\or\ 
	\or Warum arbeiten Wirtschaftler so gerne mit Excel? -- Weil sich Fehler dort viel leichter verschleiern lassen!
	\or Geht ein Neutron in die Disco, sagt der Türsteher: \enquote{Sorry, heute nur für geladene Gäste!}
	% Seite 90
	\or Kommt ein Vektor zur Drogenberatung: \enquote{Hilfe, ich bin linear abhängig!}
    \or Was ist das Schlimmste, das einem Wirtschaftsinformatiker passieren kann? -- Ein Meeting ohne WiFi.
	\or Der Computer löst Probleme, die man ohne ihn nicht hätte.
	\or Die Mengenoperation $\setminus$ ist so charmant, sie macht mir immer so liebe Komplemente.
    \or Wie nennt man einen glücklichen Statistiker? -- Ein Ausreißer.
	\or Warum werden bei BMW neuerdings keine Mathematikerinnen mehr beschäftigt? Die haben allgemein ein Auto mit $n$ Rädern konstruiert und erst danach den Spezialfall $n=4$ betrachtet.
    \or Was macht ein Wirtschaftsinformatiker, wenn er gestresst ist? -- Er erstellt ein Pivot-Table zur Analyse seiner Emotionen.
    \or Ein Programmierer stirbt und wird Petrus vorgeführt. Dieser meint: \enquote{Du warst kein guter Mensch, aber auch nicht wirklich schlecht. Du darfst dir selbst aussuchen, ob du in den Himmel oder in die Hölle kommst!} Darauf meint der Programmierer: \enquote{Ok, zeig mir den Himmel.} Petrus führt ihn zu einem riesigen Raum, randvoll mit den schnellsten Servern, neuester Hardware und den besten Workstations. \enquote{Du wirst hier programmieren}, meint Petrus. \enquote{Und wie sieht die Hölle aus?} fragt der Programmierer. Petrus antwortet darauf: \enquote{Die ist auch hier, aber dann bist du der Admin.}
	\or Was berechnen Topologen an Weihnachten? Ho-ho-homotopiegruppe.
	\or Was ist Pi? Mathematiker: \enquote{\pi ist die Zahl, die das Verhältnis vom Umfang eines Kreises und seinem Durchmesser angibt.} Physikerin: \enquote{\pi ist 3,1415927 plus/minus 0,00000005.} Ingenieur: \enquote{\pi ist ungefähr 3.}
	\or Why didn't Newton discover group theory? Because he wasn't Abel.
	\or What is a bird's favourite type of maths? Owl-gebra.
    \or Was ist das Lieblings-Wort eines Wirtschaftsinformatikers? -- Synergie: Wenn IT und Geschäftsprozesse Hand in Hand arbeiten.
    \or Wie nennt man einen Wirtschaftsinformatiker auf Diät? -- Einen Lean-Prozessberater.
	\or Wie besteigt eine Mathematikerin den Mount Everest? Sie integriert mit Hilfe einer Treppenfunktion über den Berg und steigt sie dann hinauf.
    \or Warum benutzen Wirtschaftsinformatiker immer doppelte Sicherungskopien? -- Weil sie wissen, dass man immer eine Backup-Strategie braucht.
    \or Ein Programmierer geht in ein Restaurant, hebt den Arm und sagt: \enquote{Ich werde einen Bug melden!}
	\or Wie viel ist dreimal sieben? GANZ feiner Sand! Und was ist viermal sechs? Anstrengend…
    \or Was macht ein Pirat am Computer? Er drückt die Enter-Taste.
    \or Der Roboter fragt seinen Erfinder: \enquote{Was ist mein Ziel?} -- Der Informatiker antwortet: \enquote{Effizienter arbeiten als ich.}
    \or Warum sind Informatiker so schlecht im Fußball? -- Weil sie immer versuchen, den Ball zu debuggen.
    \or Warum gehen Informatiker nicht zum Arzt? Sie googeln ihre Symptome und debuggen sich selbst!
    \or Was ist der Lieblingssnack eines Cyber Security-Experten? -- Phish-Sticks.
	\or Warum verwechseln Informatiker Weihnachten immer mit Halloween? Weil \texttt{OCT 31} gleich \texttt{DEC 25} ist.
    \or Ein Cyber-Security-Experte im Restaurant: \enquote{Entschuldigen Sie, der Käse auf meiner Pizza ist zu offen! Könnten Sie ihn besser verschlüsseln?}
    \or Drucker druckt nicht? Mal freundlich fragen, ob er lieber Schwarz-Weiß statt Farbe möchte.
    \or Warum sind Statistiker immer so optimistisch? -- Weil sie wissen, dass selbst schlechte Daten manchmal gut aussehen können.
	\or Kommt eine Mathematikstudentin in ein Fotogeschäft: \enquote{Guten Tag! Ich möchte diesen Film entwickeln lassen.} Verkäufer: \enquote{9$\times$13?} -- Studentin: \enquote{117. Wieso?} Kommt ein Mathematik-Professor in ein Fotogeschäft: \enquote{Guten Tag! Ich möchte diesen Film entwickeln lassen.} Verkäufer: \enquote{9$\times$13?} -- Professor: \enquote{Ja, das ist lösbar. Wieso?}
	\or Die Aufgaben in der Prüfung werden die gleichen wie im Kurs sein. Es werden nur die Zahlen verändert. Aber keine Sorge: Pi bleibt 3.141592…
    \or Warum gibt es keinen Konkurrenzkampf unter Wirtschaftsinformatikern? -- Weil sie alle in einem gemeinsamen Datenpool schwimmen.
    \or Kommt ein Nullvektor zum Psychiater: \enquote{Herr Doktor, ich bin immer so orientierungslos!}
	\or Der Vorlesung zum Satz von Bolzano-Weierstraß konnte ich nur zum Teil folgen.
	\or Die Ehe der Professorin soll sehr unglücklich sein, habe ich gehört! -- \enquote{Wundert mich nicht. Sie ist Mathematikerin und ihr Mann unberechenbar.}
    \or Ein Informatiker programmiert einen Roboter und fragt: \enquote{Wie fühlst du dich?} -- Der Roboter antwortet: \enquote{Null oder eins, je nach Laune.}
    \or Ein Statistiker wird gefragt: \enquote{Wieviel ist 2 + 2?} -- Antwort: \enquote{Was soll es denn sein?}
    \or Warum können Turingmaschinen keine Lieder schreiben? -- Weil sie keine Tapes mit Emotionen haben.
	\or Warum hat der Hacker die Ampel gehackt? -- Weil er den Verkehr nicht abwarten konnte!
    \or Was ist die Lieblingsband eines Informatikers? AC/DC, denn es ist wichtig, den Stromfluss zu verstehen.
    \or Warum sind Informatiker schlecht im Verstecken spielen? Weil sie immer gefunden werden, wenn sie ihre Spuren im Cache hinterlassen.
	\or Es gibt 10 Sorten von Menschen. Die einen verstehen Binärcode, die anderen nicht.
	% Seite 30
    \or Was bedeutet die Abkürzung \enquote{www}? Weltweites Warten.
	\or Mathematiker sterben nicht, sie verlieren nur einige ihrer Funktionen.
    \or Ein Hacker bestellt einen Kaffee: \enquote{Ohne Zucker, aber bitte verschlüsselt.}
	\or Ein Mathematiker ist ein Gerät, welches Kaffee in Behauptungen umwandelt.
	\or Wenn der Rechner zu heiß läuft, legen Sie einfach einen Eisbeutel drauf!
	\or Prüfer: \enquote{Malen Sie doch mal eine Skizze vom Sinus.} (Prüfling malt.) Prüfer: \enquote{Sieht doch schon ganz gut aus.} Student: \enquote{Nein, das sollte die x-Achse sein, ich bin so aufgeregt.}
    \or Was macht ein Hacker, wenn er verliebt ist? -- Er schickt eine verschlüsselte Nachricht mit einem Heartbeat.
    \or Wie nennt man einen traurigen Statistiker? -- Unter der Standardabweichung.
	\or Wo sitzt ein stalkender Graphentheoretiker? Auf einem Spannbaum.
    \or Warum bevorzugen Programmierer den Dark Mode? Weil Licht Bugs anzieht!
	\or Wie viele Quantenmechaniker braucht man, um eine Glühbirne zu wechseln? Man braucht einen Quantenmechaniker, um die Glühbirne wahrscheinlich zu wechseln.
    \or Eine Statistikerin wird gefragt, wo sie begraben werden will. Seine Antwort: \enquote{In Jerusalem, da ist die Auferstehungswahrscheinlichkeit am größten.}
    \or Wie viele Informatiker braucht man, um ein Passwort zurückzusetzen? Keinen. Das ist ein Problem für den User-Support!
    \or Warum sind BWLer so gute Verkäufer? -- Weil sie gelernt haben, dass das Wichtigste nicht das Produkt ist, sondern der Pitch!
	\or Gespräch zweier Informatikerinnen: \enquote{Wie ist denn das Wetter bei euch?} -- \enquote{Caps Lock.} -- \enquote{Hä?} -- \enquote{Na ja, Shift ohne Ende!}
	\or Was ist 2$\pi$ in der Früh? Morgentau.
    \or Warum war der Computer kalt? Das Windows wurde offen gelassen!
    \or Vier singende Informatiker bilden einen … Quad-Chor.
	\or Was sagt ein arbeitsloser Mathematiker zu einem Mathematiker, der gerade Arbeit gefunden hat? \enquote{Einmal Pommes mit Mayo bitte!}
    \or Kommt ein Suchmaschinenoptimierer in eine Bar, Kneipe, Spelunke, Wirtshaus, Gaststätte, Schänke, Bierstube…
	\or Welches Tier kann addieren? Na, ein Oktoplus!
    \or Warum gehen Statistiker selten ins Casino? -- Weil die Wahrscheinlichkeiten gegen sie sprechen.
    \or Ein Wirtschaftsinformatiker beim Arzt: \enquote{Mir tut der Kopf weh.} -- \enquote{Haben Sie schon versucht, ihn neu zu starten?}
	\or Eine Mathematikerin ist kurz davor, das erste Mal mit einem Flugzeug zu fliegen. Sie hat wahnsinnig viel Angst -- es könnte ja eine Bombe an Bord sein. Dann hat die Mathematikerin eine Idee: Sie nimmt selbst eine Bombe mit. Die Wahrscheinlichkeit, dass zwei Bomben in einem Flugzeug sind, ist wesentlich geringer, als dass eine Bombe im Flugzeug ist.
    \or Ein Informatiker im Restaurant: \enquote{Ich hätte gerne eine Pizza Margherita, ohne Bugs bitte.}
    \or Wie nennt man eine fehlerhafte Schleife in der Informatik? -- Eine Endlosschleife des Schreckens.
	\or Wie fängt ein Mathematiker in der Wüste einen Löwen? Er baut einen Käfig, setzt sich rein und definiert: 'Hier ist außen!'
    \or Ein Informatiker geht campen und stellt fest, dass das Zelt keinen Strom hat: \enquote{Fehler im System, Neustart erforderlich.}
    \or Wenn Baumeister Gebäude so bauten, wie Programmierer Programme entwickeln, dann würde der erste Specht, der vorbeikäme, die Zivilisation zerstören!!!
    \or Ein Virus kommt in eine Bar und bestellt ein Getränk. Der Barkeeper sagt: \enquote{Sorry, wir servieren hier keine Malware!}
    \or Ich habe alle meine Passwörter in \enquote{warnichtkorrekt} geändert, so sagt mir mein Rechner, wie es lautet, wenn ich es vergessen habe.
    \or Ein Statistiker auf einer Party: \enquote{Die Wahrscheinlichkeit, dass ich Spaß habe, ist verschwindend gering.}
	\or Für den Beweis dieses Satzes brauchen wir zwei Hilfssätze. Welch Dilemma.
    \or Was ist der Unterschied zwischen einem klassischen Informatiker und einem Wirtschaftsinformatiker? -- Der Wirtschaftsinformatiker optimiert den Algorithmus, während der klassische Informatiker ihn programmiert.
    \or Gespräch zweier Informatikerinnen: \enquote{Wie ist denn das Wetter bei euch?} -- \enquote{Caps Lock.} -- \enquote{Hä?} -- \enquote{Na ja, Shift ohne Ende!}
	\or Wer hat die Differentialgeometrie erfunden? -- Manni G. Faltigkeit.
	\or Ein Mathematikstudent kommt mit einem nagelneuen Fahrrad in die Uni gefahren. Sofort fragen ihn seine Kommilitonen, woher er es hat. \enquote{Ich fahre so durch den Park, als plötzlich ein Mädchen von ihrem Fahrrad springt, sich auszieht und meint, ich könne alles von ihr haben.} Darauf seine Mathe-Kommilitonen: \enquote{Echt gute Wahl, die Klamotten hätten Dir sowieso nicht gepasst!}
    \or Letzter Wunsch des Programmierers: \enquote{Bitte ein Bit.}
	\or Was ist organisierte algebraische Kriminalität? -- Ein Verbrecherring.
    \or Warum sind Wirtschaftsinformatiker so gute Projektmanager? -- Sie wissen, wie man Deadlines umprogrammiert.
    \or Ein Wirtschaftsinformatiker geht in den Supermarkt: \enquote{Haben Sie diese Produkte auch in der Cloud?}
	\or Ihr Drucker druckt nicht? Schon mal versucht, ihn zu loben?
    \or Warum können Programmierer Weihnachten und Halloween nicht auseinanderhalten? Weil \texttt{OCT 31} == \texttt{DEC 25}.
    \or IT-Support: \enquote{Machen Sie mal bitte alle Fenster zu.} User: \enquote{Auch das im Bad?}
    \or Warum lieben es Informatiker zu programmieren? -- Weil es die einzige Möglichkeit ist, mit Computern zu reden, ohne dass sie widersprechen.
    \or Warum sind Wirtschaftsinformatiker immer so ruhig? -- Sie wissen, dass der wahre Fehler zwischen Tastatur und Stuhl sitzt.
    \or Was ist der Unterschied zwischen einem Informatiker und einem Physiker? Der Physiker glaubt, ein Kilobyte sind 1000 Byte. Der Informatiker glaubt, ein Kilometer sind 1024 Meter.
    \or Warum nimmt eine Turingmaschine immer ein Band mit in den Urlaub? -- Damit sie immer was zu verarbeiten hat.
    \or Was sagt eine Turingmaschine, wenn sie aufhört zu arbeiten? -- \enquote{Ich bin jetzt halting.}
	\or Während der Vorlesung soll ein Mathematikprofessor einmal auf die schwierige Aufgabe 7$\times$9 gestoßen sein. Er bittet die Studierenden um Hilfe. Einer ruft: \enquote{62}, eine andere \enquote{65}. Darauf der Professor: \enquote{Aber das ist doch unmöglich! 7$\times$9 kann doch nur 62 ODER 65 sein!}
    \or Warum sind Statistiker schlecht in der Küche? -- Sie können die Mittelwerte nicht von den Extremen unterscheiden.
    \or Wie schreibt ein Wirtschaftsinformatiker einen Liebesbrief? -- Mit Diagrammen zur Analyse der emotionalen Gewinnchancen.
	\or Ich habe kürzlich versucht, mein Passwort zu ändern. Aber der Computer sagte, es müsse mindestens 18 Zeichen lang sein und ein Sonderzeichen enthalten. Also habe ich es in \enquote{Schneewittchen und die sieben Zwerge laufen Marathon!?} geändert.
	\or Eine Physikerin untersucht die Fallgeschwindigkeit eines Thermometers. Sie lässt ein Thermometer und ein Wachslicht gleichzeitig fallen und bemerkt, dass beide gleichzeitig unten ankommen. Schlussfolgerung: Das Thermometer fällt mit der Geschwindigkeit von Licht.
    \or Warum mögen Cyber-Security-Experten keine Witze? -- Weil sie alle Schwachstellen kennen.
	% Seite 20	
	\or Eine Soziologin, ein Physiker und eine Mathematikerin fahren im Zug. Sie schauen aus dem Fenster und sehen ein schwarzes Schaf. Soziologin: \enquote{Hier gibt es schwarze Schafe.} Physiker: \enquote{Falsch. Hier gibt es mindestens ein schwarzes Schaf.} Mathematikerin: \enquote{Immer noch falsch. Hier gibt es mindestens ein Schaf, das auf mindestens einer Seite schwarz ist.}
    \or Warum haben Turingmaschinen keine Gefühle? -- Weil sie nur in Zuständen denken können.
	\or Wenn du einen Mathematiker wählen lässt zwischen einem Brötchen und ewiger Seligkeit, was nimmt er? Natürlich das Brötchen: Nichts ist besser als ewige Seligkeit und ein belegtes Brötchen ist besser als nichts.
    \or Ein Statistiker und ein Informatiker werden gefragt, was 1+1 ergibt. Der Informatiker antwortet: \enquote{2.} Der Statistiker: \enquote{Irgendwas zwischen 1 und 3, mit 95 Prozent Konfidenz.}
    \or Ein Informatiker sagt: \enquote{Ich habe eine Turingmaschine gebaut!} -- Der andere antwortet: \enquote{Cool, aber kann sie Kaffee kochen?} -- \enquote{Ja, wenn du die richtigen Instruktionen hast.}
    \or Warum sind Roboter schlechte Künstler? -- Weil sie nur pixelgenau arbeiten können.
    \or Ein Cyber-Security-Experte auf der Party: \enquote{Kannst du dich identifizieren?} -- \enquote{Ja, ich bin sicher!}
	\or Wie nennt man ein Startup ohne Businessplan? -- Ein Studienprojekt.
    \or Ich habe eine/n Informatiker/in nach der Telefonnummer gefragt … ich bekam eine Schätzung.
    \or Warum sind Hacker schlecht im Fußball? -- Weil sie ständig die Abwehr umgehen wollen.
    \or Wie nennt man einen Berater, der keinen Rat gibt? -- Einen Influencer.
	\or Es gibt drei Sorten von Menschen: Solche, die bis drei zählen können, und solche, die nicht bis drei zählen können.
	% Seite 60	
	\or Ein Mathelehrer steht vor der Klasse und erklärt: \enquote{Es gibt keine größere und keine kleinere Hälfte. Aber warum erzähl ich euch das überhaupt, die größere Hälfte von euch versteht das ja doch nicht.}
	\or Mitten im mathematischen Vortrag erhebt einer der Anwesenden die Hand und sagt: \enquote{Ich habe zu dem, was Sie hier erzählen, ein Gegenbeispiel!} Darauf die Vortragende: \enquote{Egal, ich habe zwei Beweise!}
	\or Warum schauen Mathematiker so gerne Seifenopern im Fernsehen? Es gibt jeden Tag eine neue Folge.
	\or Wie nennt man eine leichte Kurvendiskussion? -- Banalysis.
    \or Warum haben Roboter immer so gute Laune? -- Weil sie keine Bugs, sondern Features haben!
    \or Warum mögen Wirtschaftler keine runden Tische? -- Weil es keinen klaren Vorsitzenden gibt.
	\or Party im Raum der stetigen Funktionen. Sinus und Cosinus tanzen wild auf und ab, die Polynome bilden einen Ring. Alle anwesenden Funktionen amüsieren sich prächtig, nur $e^x$ steht alleine in der Ecke. Da kommt $x^2$ vorbei und meint: \enquote{Mensch, jetzt integrier dich doch einfach mal.} $e^x$ darauf traurig: \enquote{Hab ich ja schon, aber das hat auch nix geändert.}
	\or Jede natürliche Zahl ist interessant, denn angenommen es gäbe uninteressante natürliche Zahlen. Dann gäbe es auch eine kleinste uninteressante Zahl, und das machte diese Zahl furchtbar interessant!

    \or Was bekommst du, wenn eine Spinne über deinen Monitor rennt? Eine Webseite.
	\or Studentin: \enquote{Herr Professor, können Sie uns zu diesem Beweis auch ein Beispiel vorrechnen?} Professor: \enquote{Mit diesem Beweis habe ich Ihnen bereits alle Beispiele vorgerechnet.}
	\or Wie ich 2, 3, 5, 7, 11,… finde? Prima!
    \or \enquote{Wir haben hier keine Zeit zu lesen, was auf dem Bildschirm steht!} – Ausrede eines Anwenders, der Probleme mit der Bedienung eines Programms hatte.
    \or Warum machen Statistiker keine guten Schätzungen? -- Weil sie immer den Standardfehler berechnen wollen.
    \or Warum sind Roboter immer so pünktlich? -- Weil sie keinen Spielraum für Verspätungen in ihrem Code haben.
	\or If debugging is the process of removing bugs, then programming must be the process of putting them in.
    \or Wie trinken Informatiker ihren Kaffee? \texttt{\#000000}
	\or Eine Statistikerin wird gefragt, wo sie begraben werden will. Seine Antwort: \enquote{In Jerusalem, da ist die Auferstehungswahrscheinlichkeit am größten.}
    \or Warum verlassen Wirtschaftsinformatiker nie das Haus ohne Laptop? -- Weil sie immer bereit sein müssen, Prozesse zu optimieren.
	\or Kommt ein Nullvektor zum Psychiater: \enquote{Herr Doktor, ich bin immer so orientierungslos!}
    \or Warum sind Cyber-Security-Experten schlechte Gärtner? -- Sie versuchen immer, die Pflanzen zu patchen.
	\or Was ist gelb, krumm, normiert und vollständig? Ein Bananachraum!
    \or Wie erkennt man, dass eine Turingmaschine müde ist? -- Wenn sie anfängt, sich in endlosen Schleifen zu verlieren.
    \or Wie nennt man einen Roboter, der tanzt? -- Ein Algorithmus mit Rhythmus.
	\or Was ist der Lieblingsfilm einer jeden Mathematikerin? Das Schweigen der Lemma.
    \or Wenn der Monitor nichts mehr anzeigt, liegt es wahrscheinlich daran, dass die Pixel alle Pause machen.
    \or Warum mögen Informatiker keine Partys? -- Weil sie nicht auf Bugs im sozialen Netzwerk vorbereitet sind.
	\or ++ Hinrichtung: Mathematikerin stirbt bei Beweis von Äquivalenz ++
    \or Warum können Roboter nicht lügen? -- Weil sie nur auf Wahrheitstabellen funktionieren.
    \or Was ist der Lieblingssatz eines Statistikers? -- \enquote{Im Mittel sind alle zufrieden.}
	\or Wie nennt man einen Hacker mit Sonnenbrille? -- Einen \enquote{phish}-Spezialisten!
    \or Ihr Computer fährt nicht hoch? Vielleicht mag er keine Morgenstunden.
    \or Warum trinken Informatiker Kaffee? -- Weil sie immer aufwachen müssen, wenn ihre Programme abstürzen.
    \or Warum ist ein Backup wie eine Versicherung? -- Du merkst erst, wie wichtig es ist, wenn du es brauchst.
    \or Ein Informatiker zeigt stolz seine neue Turingmaschine: \enquote{Sie kann alles lösen!} -- \enquote{Super, kannst du dann mein WLAN-Problem damit lösen?} -- \enquote{Nun, das könnte unentscheidbar sein.}
    \or Ein Wirtschaftsinformatiker geht in die IT-Abteilung: \enquote{Haben Sie eine Lizenz für Excel?} -- \enquote{Ich habe Excel in meiner DNA.}
    \or Anruf bei einer Hotline: Anrufer: \enquote{Ich benutze Windows …} Hotline: \enquote{Ja …?} Kunde: \enquote{Mein Computer funktioniert nicht richtig.} Hotline: \enquote{Das sagten Sie bereits.}
    \or Was ist die Lieblingsmahlzeit eines Wirtschaftsinformatikers? -- Ein Sandwich-Algorithmus: schnell, effizient und leicht verdaulich.
    \or Haben Sie schon versucht, das Problem mit einem Taschenrechner zu lösen? Vielleicht addiert sich der Fehler.
    \or Warum verwechseln Informatiker Weihnachten immer mit Halloween? Weil \texttt{OCT 31} gleich \texttt{DEC 25} ist.
    \or Der kürzeste Programmiererwitz: Gleich bin ich fertig!
	\or Was ist der Unterschied zwischen einem Mathematik- und einem Informatikstudenten? Der Mathestudent wollte Mathe studieren.
    \or Was passiert, wenn ein Roboter traurig ist? -- Er hat einfach nur eine schlechte Code-Zeile erwischt.
	\or Zwei Folgenglieder haben ein Date und nähern sich mit zunehmender Zeit immer mehr einander an. Da ergreift das eine die Initiative und fragt: \enquote{Voulez-vous Cauchy avec moi?}
    \or Was sagt ein Informatiker, wenn er auf die Welt kommt? \enquote{Hallo Welt!}
	\or Treffen sich zwei Pointer auf dem Stack. Sagt der eine zum anderen: \enquote{Hör auf, auf mich zu zeigen!}
    \or Warum war der Hacker traurig? -- Weil seine Freunde ihn gepatcht haben.
	\or Warum konnten Seeräuber keine runden Kanonenkugeln herstellen? Na, weil sie Pi raten!
    \or Warum sind Wirtschaftsinformatiker gut im Multitasking? -- Weil sie gleichzeitig Kosten analysieren und die Cloud optimieren können.
    \or Ein Informatiker schiebt einen Kinderwagen durch den Park. Kommt ein älteres Ehepaar: \enquote{Junge oder Mädchen?} Informatiker: \enquote{Richtig!}
    \or Was isst ein Informatiker abends beim Fernsehen? – Mikrochips.
	% Seite 10
	\or Ein Statistiker ertrinkt in einem Fluss, der im Durchschnitt nur einen Meter tief ist.
	
    \or Wenn der Server zu heiß läuft, stellen Sie ihm einfach einen Ventilator daneben.
	\or Was ist denn mit Deiner süßen kleinen Freundin, der Mathematikerin? -- \enquote{Die habe ich verlassen. Ich rufe sie an -- da erzählt sie, dass sie im Bett liegt und sich mit 3 Unbekannten rumplagt…}
    \or Ein Informatiker erklärt seinem Kind: \enquote{Wenn du auf das Spielzeug klickst, startet es. Aber pass auf, der Bug könnte es abstürzen lassen.}
	\or Frau Meier will ihrer Nachbarin zeigen, wie toll ihr Sohn Fritz schon rechnen kann: \enquote{Fritz, was ist drei mal vier?} -- \enquote{Zehn!} -- \enquote{Sehen Sie, nur um eins verrechnet!}

	\else
	Too many pages... ;)
	\fi
}

\documentclass[a4paper]{scrartcl}
\usepackage[margin=20mm]{geometry}
\usepackage{qrcode}
\usepackage{tabularray} % <— wichtig

% Link einmal definieren
\newcommand{\DriveLink}{https://drive.google.com/drive/folders/1Pqqn5CicX540Pdtz4d6UKSQVwXJDEsOY?usp=sharing}

% Zelleninhalt: Titel + QR
\newcommand{\QRcell}{%
  \begin{minipage}[c]{\linewidth}\centering
    \large{\textbf{Ersti-Heft}}\\[3mm]
    \qrcode[height=60mm]{\DriveLink}% Größe hier feinjustieren (60–70 mm)
  \end{minipage}%
}

\begin{document}
\thispagestyle{empty}

\begin{center}
  \begin{tblr}{
    colspec = {X[c] X[c]},  % 2 zentrierte Spalten, gleiche Breite
    colsep  = 14mm,         % horizontaler Abstand zwischen Spalten
    rowsep  = 14mm,         % vertikaler Abstand zwischen Zeilen
    stretch = 0             % keine automatische Streckung
  }
    \QRcell & \QRcell \\
    \QRcell & \QRcell \\
    \QRcell & \QRcell
  \end{tblr}
\end{center}

\end{document}

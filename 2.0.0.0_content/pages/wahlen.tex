\section{Wahlen}

\subsection{Gremienwahlen}
An der Hochschule Bonn-Rhein-Sieg finden regelmäßig Wahlen für die studentische Selbstverwaltung statt.  
Studierende können wählen und selbst kandidieren.

\subsection*{Studentische Selbstverwaltung}
\begin{itemize}[noitemsep,nolistsep]
  \item \textbf{Fachschaftsräte} – gewählte Vertretung auf Fachbereichsebene nach § 53 HG NRW.  
  Sie vertreten die Interessen der Studierenden im Fachbereich, wirken in Gremien mit und organisieren Angebote wie Erstsemester-Einführungen, Lernmaterialien oder Veranstaltungen.  
  Für dich sind sie die direkten Ansprechpartner:innen im Studienalltag, z. B. bei Fragen zu Modulen, Problemen mit Prüfungen oder der Suche nach Mitstudierenden für Projekte oder leiten an die richtige stelle weiter.

  \item \textbf{Studierendenparlament (StuPa)} – nach § 53 HG NRW das \emph{höchste beschlussfassende Organ} der Studierendenschaft.  
  Es entscheidet über die Verwendung der Semesterbeiträge in Ordnugnen, erlässt Satzungen und legt die Grundlinien studentischer Hochschulpolitik fest.  
  Als Student:in profitierst du davon, dass über Finanzen (z. B. Semesterticket, Kulturförderung) transparent und demokratisch entschieden wird und deine Interessen auf Hochschulebene gebündelt werden.  

  \item \textbf{Allgemeiner Studierendenausschuss (AStA)} – die Exekutive der Studierendenschaft.  
  Er setzt die Beschlüsse des StuPa um, vertritt die Studierenden nach außen und bietet konkrete Services wie das Semesterticket, Beratungsangebote (Soziales, Recht, Internationales) oder Kultur- und Sportveranstaltungen.  
  Für dich bedeutet das: Der AStA ist deine Service- und Beratungsstelle, die dir bei Alltagsproblemen hilft und dein Studium durch zusätzliche Angebote bereichert.  
\end{itemize}

\subsection*{Studentische Wahlen im Detail}
\begin{itemize}[noitemsep,nolistsep]
  \item \textbf{Fachschaftswahlen}
  Jeder Fachbereich hat eine Fachschaft(alle volleingeschriebenen Studierenden) und diese einen Fachschaftsrat, die die Ersti-Woche, Events, Partys und Beratungen organisiert und deine Interessen vertritt.  
  \textbf{Wahlen:} jährlich; alle Studierenden des Fachbereichs dürfen wählen und kandidieren.

  \item \textbf{StuPa-Wahlen}
  Das \textbf{Studierendenparlament (StuPa)} entscheidet über Finanzen durch den Haushaltsauschuss, studentische Projekte und Positionen gegenüber der Hochschule.  
  \textbf{Wahlen:} jährlich; wahlberechtigt sind alle Studierenden der Hochschule.

  \item \textbf{AStA}
  Der \textbf{Allgemeine Studierendenausschuss (AStA)} setzt StuPa-Beschlüsse um, vertritt die Studierenden nach außen und bietet Services.  
  \textbf{Wahl:} nicht direkt; das StuPa wählt den AStA nach jeder Wahlperiode neu.
\end{itemize}

\subsection*{Wahlturnus}
\begin{itemize}
  \item Studentische Gremien: jährlich  
\end{itemize}


\subsection*{Warum wählen?}
Mit deiner Stimme entscheidest du, wer deine Interessen gegenüber Hochschule und Politik vertritt.  
Ob Semesterticket, Prüfungsordnungen oder Beratungsangebote, überall stecken gewählte Studierende dahinter.

\vspace{2em}

\subsection{Hochschulwahlen}
Hochschulwahlen betreffen die akademische Selbstverwaltung.  
Hier werden z. B. Vertreter:innen für den \textbf{Senat}, die \textbf{Fachbereichsräte} und Studierendenplätze im \textbf{Hochschulrat} bestimmt.  
Diese Gremien treffen Entscheidungen zu Lehre, Studium, Prüfungsordnungen und der strategischen Ausrichtung der Hochschule.

\subsection*{Akademische Selbstverwaltung}
\begin{itemize}
  \item \textbf{Senat} – das zentrale Gremium der Hochschule.  
  Er entscheidet über die Prüfungsordnungen, Studienordnungen und alle Grundsatzfragen von Studium und Forschung.  
  Im Senat sind auch Studierende vertreten, sodass du direkt Einfluss auf hochschulweite Regelungen und Entwicklungen nehmen kannst.  

  \item \textbf{Fachbereichsräte} – die Parlamente der einzelnen Fachbereiche.  
  Sie beschließen über Lehr- und Forschungsinhalte, die Organisation der Studiengänge und die Verteilung von Ressourcen im Fachbereich.  
  Für dich bedeutet das: Hier kannst du über die Ausgestaltung deines Studiums und die Weiterentwicklung deines Fachs mitentscheiden.  

  \item \textbf{Hochschulrat} – das strategische Kontroll- und Beratungsgremium der Hochschule.  
  Er überwacht die wirtschaftliche Lage, übt die Aufsicht über die Geschäftsführung aus, beschließt den Entwicklungsplan und berät Präsidium und Senat in Grundsatzfragen.  
  Auch wenn der Hochschulrat eher auf einer übergeordneten Ebene arbeitet, können diese an der strategische Ausrichtung der Hochschule mitgestalten.  
\end{itemize}


\subsection*{Wahlturnus}
\begin{itemize} 
  \item Senat und Fachbereichsräte: alle zwei Jahre  
  \item Hochschulrat: alle fünf Jahre (Bestellung, keine klassische Wahl)  
\end{itemize}


\subsection*{Warum wählen?}
Mit deiner Stimme entscheidest du, wer deine Interessen gegenüber Hochschule und Politik vertritt.  
Ob Semesterticket, Prüfungsordnungen oder Beratungsangebote, überall stecken gewählte Studierende dahinter.


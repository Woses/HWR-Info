\section*{WLAN an der H-BRS}

\begin{enumerate}[label=\arabic*., noitemsep,nolistsep]
    \item \textbf{Eduroam via EasyRoam}
    EasyRoam ist eine App/Software des \textbf{DFN (Deutsches Forschungsnetz)} für iOS, Mac, Windows, Android und Linux (Arch Linux über AUR-Paket).  
    \textbf{So einfach geht's:}
    \begin{enumerate}[label*=\arabic*., left=1em]
        \item App öffnen oder über \href{https://www.easyroam.de}{easyroam.de} nach \textit{"Hochschule Bonn-Rhein-Sieg IDM"} suchen.
        \item Login über \textbf{MIA SSO/IDP}.
        \item Neues Zertifikat anfordern oder altes ersetzen.
        \item Mit \textit{eduroam} verbinden. Achtung: manche Netzwerke heißen abweichend, z.B. \textit{eduroam-cs, eduroam-math, eduroam-stw, eduroam-ukb}. Dann manuell einrichten oder \href{https://cat.eduroam.org/}{CAT-Tool} nutzen.
        \item \textbf{Tipps:} In stark frequentierten Räumen (z.B. Hörsäle) kann Eduroam langsam sein. Lade Skripte/Folien vorab herunter, z.B. in unserem Büro.
    \end{enumerate}
    \textbf{Anleitungen vom DFN:}
    \begin{itemize}[noitemsep,nolistsep]
        \item \href{https://doku.tid.dfn.de/de:eduroam:easyroam#die_webseite_wwweasyroamde_fuer_die_eduroam_nutzenden}{EasyRoam Portal}
        \item \href{https://doku.tid.dfn.de/de:eduroam:easyroam#installation_der_eduroam_profile_auf_w10_11_mit_der_easyroam_app}{Windows 10/11}
        \item \href{https://doku.tid.dfn.de/de:eduroam:easyroam#installation_der_eduroam_profile_auf_macos}{macOS}
        \item \href{https://doku.tid.dfn.de/de:eduroam:easyroam#installation_der_eduroam_profile_auf_watchos}{watchOS}
        \item \href{https://doku.tid.dfn.de/de:eduroam:easyroam#installation_der_eduroam_profile_auf_ios_12}{iOS~$\leq$~12}
        \item \href{https://doku.tid.dfn.de/de:eduroam:anleitungen:easyroamapp#installation_der_eduroam_profile_mit_der_easyroam_app_im_app_store_auf_ios_13}{iOS~$\geq$~13}
        \item \href{https://doku.tid.dfn.de/de:eduroam:easyroam:installation_der_easyroam_profile_auf_chromeos_geraeten}{ChromeOS}
        \item \href{https://doku.tid.dfn.de/de:eduroam:anleitungen:easyroamapp#installation_der_eduroam_profile_mit_der_easyroam_app_auf_android_10}{Android~$\geq$~10}
        \item \href{https://doku.tid.dfn.de/de:eduroam:easyroam#installation_der_eduroam_profile_auf_android_9}{Android~$\leq$~9}
        \item \href{https://doku.tid.dfn.de/de:eduroam:easyroam#installation_der_easyroam_app_auf_linux_geraeten_network_manager}{Linux (Network Manager)}
        \item \href{https://doku.tid.dfn.de/de:eduroam:easyroam#installation_der_easyroam_profile_auf_linux_geraeten}{Linux allgemein}
        \item \href{https://doku.tid.dfn.de/de:eduroam:easyroam#installation_der_easyroam_profile_auf_linux_geraeten_ohne_desktop_umgebung_wpa-supplicant_only}{Linux ohne Desktop (wpa-supplicant)}
    \end{itemize}
    \textbf{Schritt-für-Schritt Anleitungen vom Fachbereich:}
    \begin{itemize}[noitemsep,nolistsep]
        \item \href{https://faq.infcs.de/wlan-eduroam/eduroam-easyroam/mac/}{Mac}
        \item \href{https://faq.infcs.de/wlan-eduroam/eduroam-easyroam/android/}{Android}
        \item \href{https://faq.infcs.de/wlan-eduroam/eduroam-easyroam/windows/}{Windows}
        \item \href{https://faq.infcs.de/wlan-eduroam/eduroam-easyroam/linux/}{Linux}
    \end{itemize}
    \item \textbf{FS-INF-AP}  
    WLAN-Zugang über unseren Fachschats-AP:  
    \begin{itemize}[noitemsep,nolistsep]
        \item Zum Testen von Eduroam, VPN oder Proxy-Zugang.
    \end{itemize}
\end{enumerate}

\begin{center}
    \item \qrcode[height=30mm]{WIFI:T:WPA;S:FS-INF-AP;P:internyet;;}
    \item SSID: \texttt{FS-INF-AP}, Passwort: \texttt{internyet}
\end{center}


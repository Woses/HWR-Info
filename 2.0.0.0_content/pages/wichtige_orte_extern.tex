\section{Wichtige Orte Extern}

\subsection{Studierendenwerk Bonn}

\subsubsection{Psychologische Beratungsstelle (PBS)}
\vspace{-4mm}
Wenn dir im Studium mal alles zu viel wird, du mit Ängsten, Überforderung oder persönlichen Problemen kämpfst, kannst du dich an die \textbf{Psychologische Beratungsstelle (PBS)} des Studierendenwerks Bonn wenden.  
Die Gespräche sind \textbf{vertraulich, kostenlos} und richten sich speziell an Studierende der H-BRS.  
Es gibt Einzelgespräche, offene Sprechstunden sowie Online- oder Telefontermine.  

\textbf{Adresse:} Adenauerallee 63, 53113 Bonn
\textbf{Standort Sankt Augustin:} Raum E221 (Termine vorher online einsehen) \\
\qrcode[height=20mm]{https://www.studierendenwerk-bonn.de/beratung-soziales/psychologische-beratungsstelle-pbs}
\href{https://www.studierendenwerk-bonn.de/beratung-soziales/psychologische-beratungsstelle-pbs}{https://www.studierendenwerk-bonn.de/beratung-soziales/pbs}
\vspace{-4mm}

\subsubsection{Schreibberatung}
\vspace{-4mm}
Wenn du Unterstützung beim Schreiben deiner Haus- oder Abschlussarbeit brauchst oder Hilfe im Umgang mit Schreibblockaden suchst, kannst du die \textbf{Schreibberatung} nutzen.  
Du bekommst individuelles Feedback zu deinem Schreibprozess und zur Struktur deiner Texte.  
Es gibt auch eine \emph{offene Schreibberatung} in der Universitäts- und Landesbibliothek.  

\textbf{Adresse:} Adenauerallee 63 / 39-41, 53113 Bonn \\
\qrcode[height=20mm]{https://www.studierendenwerk-bonn.de/beratung-soziales/schreibberatung}
\href{https://www.studierendenwerk-bonn.de/beratung-soziales/schreibberatung}{https://www.studierendenwerk-bonn.de/beratung-soziales/schreibberatung}
\vspace{-4mm}

\subsubsection{Finanzielle Hilfe}
\vspace{-4mm}
\paragraph{BAföG}
Wenn du dein Studium mit \textbf{BAföG} finanzieren möchtest, ist das Studierendenwerk Bonn deine Anlaufstelle.  
Dort bekommst du Unterstützung bei der Antragstellung, bei Fragen zur Förderhöhe, Anrechnung von Einkommen, Auslandsförderung oder Hilfe zum Studienabschluss.  

\textbf{Adresse:} Lennéstraße 3, 53113 Bonn \\
\qrcode[height=20mm]{https://www.studierendenwerk-bonn.de/finanzieren}
\url{https://www.studierendenwerk-bonn.de/finanzieren}




\clearpage
\paragraph{Freitisch}

Falls es mit dem Geld knapp ist, kannst du mit dem \textbf{Freitisch} kostenlos in der Mensa essen.  
Montag bis Freitag (außer an Feiertagen) gibt es morgens belegte Brötchen und ein Getränk, mittags ein Hauptgericht mit Beilagen und Getränk (auch Food-Truck möglich), und nachmittags ein Angebot aus der Cafeteria.  
\begin{flushright}
  \url{https://www.studierendenwerk-bonn.de/beratung-soziales/freitisch}
  \qrcode[height=20mm]{https://www.studierendenwerk-bonn.de/beratung-soziales/freitisch}
\end{flushright}



\paragraph{Darlehen und Kredite}

Neben BAföG kannst du mit Unterstützung des BAföG-Amts auch auf Darlehen und Kredite zurückgreifen:  

\begin{itemize}[noitemsep,nolistsep]
  \item Daka-Darlehen
  \item BMBF-Bildungskredit
  \item Hilfe zum Studienabschluss
\end{itemize}
\begin{flushright}
  \url{https://www.studierendenwerk-bonn.de/finanzieren/studienkredite}
  \qrcode[height=20mm]{https://www.studierendenwerk-bonn.de/finanzieren/studienkredite}
\end{flushright}

\paragraph{Stipendien}

\textbf{Stipendien} sind finanzielle Förderungen, die du nicht zurückzahlen musst.  
Sie werden nicht nur für gute Noten vergeben, sondern auch für Engagement, besondere Lebensumstände oder deinen Werdegang.  
Ein Stipendium kann dir den finanziellen Druck nehmen und neue Freiräume schaffen.  


\begin{itemize}[itemsep=3pt,parsep=0pt,topsep=0pt]
  \item \url{https://www.stipendiumplus.de} – 13 große Begabtenförderwerke
  \item \url{https://www.stipendienlotse.de} – offizielle Datenbank vom BMBF
  \item \url{https://www.e-fellows.net} – über 800 Stipendienprogramme
  \item \url{https://www.elfi.info} – Rechercheportal für Fördermöglichkeiten
  \item \url{https://www.auslandsstipendien.de} – speziell für Auslandsförderung
  \item \url{https://www.stiftungssuche.de} – über 10.000 Stiftungen
  \item \url{https://www.deutsches-stiftungszentrum.de} – wissenschaftsnahe Stiftungen
  \item \url{https://www.maecenata.eu} – bundesweite Stiftungsdatenbank
  \item \url{https://www.deutschlandstipendium.de} – einkommensunabhängige Förderung
  \item \url{https://www.aufstiegsstipendium.de} – für Berufserfahrene im Studium
  \item \url{https://www.arbeiterkind.de/studium-finanzieren/stipendien} – Extra für Arbeiterkinder
\end{itemize}

\par\vspace{-7mm}


\clearpage
\subsubsection{Studieren mit Kind}
\paragraph{BAföG mit Kind}

Studierende mit Kind können im Rahmen des BAföG einen \textbf{Kinderbetreuungszuschlag} erhalten: \textbf{130\,€ pro Kind unter 10 Jahren}, zusätzlich zur regulären Förderung.  

Außerdem kann BAföG auch über die übliche Altersgrenze von 30 Jahren hinaus gewährt werden, wenn die Pflege und Erziehung des Kindes ursächlich für die Überschreitung waren.  

\qrcode[height=20mm]{https://www.uni-bonn.de/de/universitaet/chancengerechtigkeit/familiengerechte-hochschule/studium-und-familie/finanzierung}
\url{https://www.uni-bonn.de/de/universitaet/chancengerechtigkeit/familiengerechte-hochschule/studium-und-familie/finanzierung}

\paragraph{Kindertagesstätte und Angebote}

Wenn du mit Kind am Campus Sankt Augustin studierst, kannst du die \textbf{Kindertageseinrichtung des Studierendenwerks} nutzen (Europaring 86). Dort werden rund 30 Kinder bis zum Schuleintritt betreut, %auch mit Plätzen für Studierende der H-BRS.  

Außerdem gibt es \textbf{Familienwohnungen} im Wohnheimen (mit Wohnberechtigungsschein) sowie den \emph{„Bonnapetit“-Kinderteller} in der Mensa, der Kindern eine kostenlose Mahlzeit ermöglicht.

Zusätzlich unterstützt dich die Hochschule mit dem Beratungsangebot \textbf{HELP}, das dir bei allen Fragen rund um Studium und Familie zur Seite steht. (Betreuung, Urlaubssemester, Eltern-Kind-Arbeitsplätze, Nothilfefonds).   

\qrcode[height=20mm]{https://www.studierendenwerk-bonn.de/studieren-mit-kind/}
\url{www.studierendenwerk-bonn.de/studieren-mit-kind/}

\vspace{7pt}
\textbf{Tipp:} Auf dem Hochschulgelänge, bzw. im Innenhof der Hochschule gibt es sogar einen Spielplatz den die Kinder von den Studierenden der H-BRS nutzen können.

\clearpage
\paragraph{Windelgeld}

Studierende der H-BRS und Universität Bonn können \textbf{einmal pro Kind unter 6 Jahren} ein Windelgeld beantragen: \textbf{200\,€} in einer Summe pro Semester.  

Anspruchsberechtigt sind Studierende mit geringem Einkommen, die keine anderen Leistungen wie BAföG (Abschnitt II), Daka-Darlehen oder KfW-Studienkredit beziehen und deren Nettoeinkommen nach Abzug von Miete und Versicherung unter einer bestimmten Schwelle liegt.  

Das Windelgeld hilft, die ersten Kosten (z.\,B. Windeln, Ausstattung) abzufedern, wenn die Finanzierung knapp ist.  

\begin{flushright}
  \url{https://www.studierendenwerk-bonn.de/beratung-soziales/windelgeld}
  \qrcode[height=20mm]{https://www.studierendenwerk-bonn.de/beratung-soziales/windelgeld}
\end{flushright}


\paragraph{Beratungsangebot - Studieren mit Kind}

Das „Studieren mit Kind“-Programm des Studierendenwerks Bonn bietet \textbf{flexible Kinderbetreuung}, unter anderem über \textbf{KinderSt.E.R.N.}, ein stundenweises Angebot am Nachmittag oder Abend.  

Zusätzlich gibt es \textbf{Kitaplätze} an mehreren Standorten (Bonn, Sankt Augustin, Rheinbach) mit ganztägiger Betreuung, teils in Kooperation mit der Universität und der Hochschule.  

Die Beratung umfasst Hilfe bei Organisation, Vereinbarkeit, Finanzierung und Betreuungsmöglichkeiten für studierende Eltern.


\begin{flushright}
  \url{www.studierendenwerk-bonn.de/studieren-mit-kind/}
  \qrcode[height=20mm]{https://www.studierendenwerk-bonn.de/studieren-mit-kind/}
\end{flushright}




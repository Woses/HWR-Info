\section{VPN, Cloud, Office 365 und Service}

\begin{enumerate}[label*=\arabic*.,noitemsep,nolistsep]
    % VPN und Proxy
    \item \textbf{VPN und Proxy}
    \begin{enumerate}[label*=\arabic*.]
        \item \textbf{Proxy für FB02 und akademische Software}  
        Zugriff auf interne Systeme des Fachbereichs ist möglich, \textbf{VPN wird empfohlen}.  
        \begin{itemize}[left=1em]
            \item Proxy-PAC: \href{http://proxypac.inf.h-brs.de/proxy.pac}{proxypac.inf.h-brs.de/proxy.pac}
            \item HTTP-Proxy: \href{http://www-cache.inf.h-brs.de:8080}{www-cache.inf.h-brs.de:8080} oder \href{http://10.20.204.15:8080}{10.20.204.15:8080}
            \item Kein Zugriff auf die Bibliothek über diesen Proxy
            \item \textcolor{blue}{Empfohlen:} iOS, Mac, Windows
            \item \textcolor{red}{Nicht empfohlen:} Android, Linux (komplizierte Einrichtung, langsame Verbindung)
        \end{itemize}

        \item \textbf{VPN Tunnel mit OpenVPN}  
        Der Fachbereich stellt einen eigenen VPN-Server bereit.  
        \begin{itemize}[left=1em]
            \item Download: \href{https://portal.inf.h-brs.de/certmk/}{Configs, Zertifikate, Schlüssel}
            \item Empfohlene Plattformen: Android, Windows, Linux, Mac
            \item Android: App \href{https://play.google.com/store/apps/details?id=de.blinkt.openvpn}{OpenVPN für Android} verwenden
            \item iOS: nicht empfohlen (nur eine VPN-Verbindung gleichzeitig; Proxy parallele Nutzung besser)
        \end{itemize}

        \item \textbf{VPN zur Bibliothek / Fernzugriff}  
        Für Bibliothekszugriff empfohlen, besonders wenn kein VPN für FB02 benötigt wird.  
        \begin{itemize}[left=1em]
            \item Proxy für Bibliothek: \href{https://www.h-brs.de/files/proxy.pac}{proxy.h-brs.de}
            \item Infos \& Anleitung: \href{https://www.h-brs.de/de/bib/fernzugriff}{Fernzugriff Info}
            \item Konfigurationen:
            \begin{itemize}
                \item Windows 10: \href{https://www.h-brs.de/de/bib/fernzugriff-konfiguration-windows-10}{Anleitung}
                \item Firefox: \href{https://www.h-brs.de/de/bib/fernzugriff-konfiguration-mozilla-firefox}{Anleitung}
                \item Mac OS: \href{https://www.h-brs.de/de/bib/fernzugriff-konfiguration-safari}{Anleitung}
                \item iPad: \href{https://www.h-brs.de/de/bib/konfiguration-ipad-fuer-den-fernzugriff}{Anleitung}
                \item FAQ: \href{https://www.h-brs.de/de/bib/haeufig-gestellte-fragen-zum-fernzugriff}{Häufig gestellte Fragen}
            \end{itemize}
        \end{itemize}
    \end{enumerate}
    % Systeme und Dienste
    \item \textbf{Systeme und Dienste}
    \begin{enumerate}[label*=\arabic*., left=1em]
        \item \textbf{Cloudspeicher}
        \begin{itemize}[left=1em]
            \item \textbf{H-BRS Sciebo} (30 GB)  
            Registrierung notwendig, ähnlich wie Nextcloud oder ownCloud.  
            \begin{itemize}
                \item Beispielkennung: \texttt{mmuste2s@h-brs.de}
                \item Weblogin: \href{https://h-brs.sciebo.de/login}{https://h-brs.sciebo.de/login}
                \item Verwaltung: \href{https://www.sciebo.de/mysciebo_uebersicht}{My.Sciebo}
                \item Dokumenteneditor für Text, Tabellen, Präsentationen
                \item Apps für PC \& Smartphone: \href{https://www.sciebo.de/de/download}{Download}
                \item Hilfe: \href{https://www.sciebo.de/de/hilfe}{Anleitung}
                \item Kontakt: \href{https://www.sciebo.de/de/kontakt/index.html}{Kontaktformular}
                \item Account gültig für 12 Monate, verlängerbar solange Mitglied einer Einrichtung
                \item Verbindung mit Nextcloud-Apps möglich
            \end{itemize}
            \item \textbf{BIB-Cloud:} 15 GB Speicher
            \begin{itemize}
                \item Account existiert solange du an der Hochschule immatrikuliert bist. Keine Erstellung notwendig
                \item Weblogin: \href{https://bib-cloud.bib.hochschule-bonn-rhein-sieg.de}{https://bib-cloud.bib.hochschule-bonn-rhein-sieg.de}
                \item Bibliotheksnummer + MIA-Passwort
                \item Verbindung mit Nextcloud-Apps möglich \url{https://nextcloud.com/}
            \end{itemize}
        \end{itemize}
    \end{enumerate}
    \item \textbf{Service}
    \begin{enumerate}
        \item \textbf{Servicepoint Fachbereich Informatik}
        \begin{itemize}[noitemsep,nolistsep]
            \item \textbf{Technische Infrastruktur:} Hilfe bei Account-Problemen, Passwörtern, WLAN, VPN und Zugängen zu Hochschuldiensten.  
            \item \textbf{Studienbetrieb:} Unterstützung bei defekten Rechnern, Raumtechnik, Beamerproblemen und Raumverleih für Lerngruppen.  
            \item \textbf{Geräte- und Materialverleih:} Adapter, Kabel, Whiteboardmarker und mehr.  
            \item \textbf{Erreichbarkeit:} Werktags 08:00–18:00 in A102.1 \\ oder per Mail an \href{mailto:servicepoint@mail.inf.h-brs.de}{servicepoint@mail.inf.h-brs.de}.  
            \item \textbf{Tipp:} Macht euch frühzeitig mit den Hochschulservices vertraut, um Stress vor Abgaben zu vermeiden.  
        \end{itemize}
        \item \textbf{LEA-Support für Informatik}
        \begin{itemize}[noitemsep,nolistsep]
            \item \textbf{Zuständigkeit:} Für den Fachbereich Informatik ist \textbf{Melanie Klöß} deine Ansprechpartnerin im E-Learning-Team. 
            \item \textbf{Service:} Sie unterstützt dich beim administrativen LEA-Support und bei allen Fragen rund um digitale Lehre und E-Prüfungen. 
            \item \textbf{Raum:} A 167 
            \item \textbf{Tel:} +49 2241 865 788
            \item \textbf{Mail:} \href{mailto:melanie.kloess@h-brs.de}{melanie.kloess@h-brs.de}
            \item \textbf{Webseite:} \url{https://www.h-brs.de/de/elearning/kontakt}.  
        \end{itemize}
        \item \textbf{Druckpunkt – Dein Copyshop an der H-BRS}
        \begin{itemize}[noitemsep,nolistsep]
            \item \textbf{Abschlussarbeiten drucken und binden} (Hardcover, Softcover, Spiralbindung) sowie Poster, Plakate und Großformate.  
            \item \textbf{Schnell und direkt auf dem Campus} – Expressdruck möglich.  
            \item \textbf{Weitere Services:} Scans, Bildbearbeitung, Geschäftspapiere, Visitenkarten.  
            \item \textbf{Kontakt:} Grantham-Allee 20, Mo–Do 8:30–16:00, Fr 8:30–15:00, 
            \item \textbf{Tel:} +49 2241 1654566, 
            \item \textbf{Webseite:} \url{https://druckpunkt-on.net/}.  
        \end{itemize}

    \end{enumerate}
\end{enumerate}


\newpage

\subsubsection{Microsoft Office 365 Education}

\href{https://info.h-brs.de/de/home}{\textbf{info.h-brs.de}}, hier findest du einfache Anleitungen, zum Beispiel für WLAN und Office.  
Speziell für Microsoft 365: \href{https://info.h-brs.de/de/extern/m365}{\textbf{info.h-brs.de/m365}} schamlos kopiert von da. 

\subsection*{Was ist Microsoft 365 Education?}
Das \textbf{Microsoft 365 Education-Paket} steht allen immatrikulierten Studierenden der Hochschule für die gesamte Dauer ihres Studiums kostenlos zur Verfügung.  

Es beinhaltet:
\begin{itemize}[itemsep=3pt, parsep=0pt, topsep=2pt]
  \item \textbf{Word}, \textbf{PowerPoint}, \textbf{Excel}, \textbf{OneNote}, \textbf{Outlook}, \textbf{Publisher} und \textbf{Access}.  
  \item Einige Anwendungen stehen ausschließlich unter Microsoft Windows zur Verfügung.  
  \item Die Produkte können auf bis zu \textbf{5 PCs oder Macs}, \textbf{5 Tablets} und \textbf{5 Smartphones} pro Benutzer installiert werden.  
\end{itemize}

Außerdem ist \textbf{OneDrive for Business} enthalten, mit \textbf{1 TB Cloud-Speicher}.  
\emph{Wichtig:} Für die Sicherung deiner Daten in der Cloud bist ausschließlich du selbst verantwortlich! Eine Datensicherung durch Microsoft oder die Hochschule findet nicht statt.  

\subsection*{Installation auf PC oder Laptop}
So installierst du Microsoft 365 Education Schritt für Schritt:

\begin{enumerate}
  \item Rufe \url{https://portal.office.com/account/} auf.  
  \item Gib deine Hochschul-E-Mail-Adresse ein (\texttt{vorname.nachname@smail.inf.h-brs.de}) und klicke auf \emph{Weiter}.  
  \item Melde dich mit deinem \textbf{MIA-Passwort} an.  
  \item Die Nachfrage, ob du angemeldet bleiben möchtest, kannst du verneinen.  
  \item Unter \emph{Mein Konto} in der Kachel \emph{Büro-Apps und -Geräte} wähle \emph{Büro installieren}.  
  \item Die Installationsdatei wird automatisch heruntergeladen. Führe diese per Doppelklick aus und folge dem Setup-Assistenten.  
\end{enumerate}

\subsection*{Installation auf Tablets und Smartphones}
Für mobile Geräte funktioniert die Installation so:  
Suche die gewünschte App (z. B. \emph{Word}) im App Store deines Endgeräts (z. B. \emph{Play Store} bei Android), installiere die Anwendung und melde dich anschließend wie in den Schritten 2–4 beschrieben mit deiner Hochschul-Mailadresse und deinem MIA-Passwort an.  


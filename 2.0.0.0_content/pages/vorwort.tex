\section*{\textbf{How to Ersti-Heft}}

\textbf{Moin und willkommen im Studium!}\\
%Die Absicht und die Idee hinter dem Heft 
Dieses Heft ist die erste Ausgabe meines Ersti-Hefts für alle Informatik-Erstis an der H-BRS.
Der Studienstart ist eine spannende Zeit – voller neuer Eindrücke, Menschen und tonnenweise Informationen. Da verliert man schnell mal den Überblick.

\vspace{-2mm}

Genau dabei soll dir dieses Heft helfen: Es gibt dir Orientierung und erleichtert dir den Einstieg ins Studium.

\vspace{-2mm}

%Dieses Heft ist nicht nur auf Papier gedacht, sondern 
Dieses Helft funktioniert am besten in Kombination mit der dazugehörigen PDF.  
Ganz ohne mühsames abtippen: Die wichtigsten Inhalte findest du direkt als \textbf{QR-Code zum Einscannen} oder als \textbf{Link} im Heft und in der PDF, um auf die jeweiligen Ressourcen zuzugreifen. Lass dich nicht von den vielen Seiten abschrecken. \\
%Das Heft ist so aufgebaut, dass du die Kapitel lesen kannst, die dich interessieren. Du musst nicht alles von vorne bis hinten durchlesen.

\underline{\textbf{Bitte beachte:}} Manche Informationen können sich nach Redaktionsschluss ändern. Dies ist leider nicht zu verhindern. In der PDF dieses Heftes und über die verlinkten Seiten stehen dir jedoch immer die \textbf{’’aktuellsten Ressourcen’’} zur Verfügung. Basierend darauf wie oft die Webseiten aktualisiert werden, kann es sein, dass du dort neuere Informationen findest als zum aktuellen Stand in diesem Heft/PDF.\\
Falls du dir nicht sicher bist, ob eine Information noch aktuell ist, schau am besten auf die verlinkte Seite oder kontaktiere die dort angegebenen Ansprechpartner.\\
Frag lieber einmal zu viel nach, als einmal zu wenig!

%Das sich der Druck verzögert erhaltst du diese PDF zur gleichen Zeit wo du das Heft erhalten hättest sollen. Ich liefere es dir nach -Timo

%\vspace{-1mm}

\textbf{Tipp 1:} Speichere die PDF auf deinem Handy oder Tablet ab, damit du sie immer griffbereit hast.
\textbf{Tipp 2:} Benutze das Inhaltsverzeichnis um schnell wo hinzukommen. Du kannst so über die PDF auch direkt zu den Kapiteln springen.\\
\textbf{Tipp 3:} \texttt{STRG+F} ist dein Freund wenn du noch effizenter sein willst beim Suchen (Falls du es nicht kennst, der Shortcut hilft um in Objekten nach Begriffen zu suchen).

\vspace{-2mm}

\medskip
\noindent
\textbf{Sicherheit der QR-Codes}  
Alle QR-Codes in diesem Heft werden automatisch im LaTeX erzeugt. Jeder Code verweist ausschließlich auf den Link, der direkt darunter, daneben oder darüber steht. So gibt es keine Verwechslungen und kein Risiko durch manipulierte QR-Codes.

Viel Erfolg und einen guten Start ins Studium wünscht dir\\
\textbf{Timo Mansfeld}

\vspace{-5mm}

\begin{center}
  %\vspace{-5mm}
  % Beispiel QR-Code mit Link zum PDf vom Heft
  \vfill
  \small \textbf{Hier bekommst du die PDF des Hefts.}\\
  \small\href{https://drive.google.com/drive/folders/1Pqqn5CicX540Pdtz4d6UKSQVwXJDEsOY?usp=sharing}{Google Drive Ersti-Heft}\\
  \vspace{1mm}
  \qrcode[height=45mm]{https://drive.google.com/drive/folders/1Pqqn5CicX540Pdtz4d6UKSQVwXJDEsOY?usp=sharing}\\
  \vfill
\end{center}


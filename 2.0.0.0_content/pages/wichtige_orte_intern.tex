\section{Wichtige Orte Intern}

Hier findest du eine Übersicht über die wichtigsten Orte auf dem Campus Sankt Augustin, die dir den Studienalltag erleichtern.
\url{https://www.h-brs.de/de/inf/ansprechpartner-anlaufstellen}

\subsection{Fachschaft A051}
Natürlich der wichtigste Ort von allen: unser Büro \textbf{A051} der Fachschaft!

Du kannst bei uns bekommen:
\begin{itemize} [noitemsep,nolistsep]
    \item Ansprechpartner:innen für alle möglichen Fragen rund ums Studium
    \item Altklausuren (sofern vorhanden) \textbf{Nur mit USB-Stick oder SSD abholbar!}
    \item Drucker (kostenlos für Studierende vom FB02 im begrenzten Umfang)
    \item Getränke für 1€
    \item Kostenloses Wassereis
    \item Eine Couch zum Entspannen
    \item Einen Ort zum Treffen mit anderen Studierenden
\end{itemize}

Du findst uns direkt an der Hochschulstraße, geradeaus vom Haupteingang. Wir haben einen Bildschrim direkt vor dem Eingang des Büros mit aktuellen Events und Aktionen und den aktuellen Bahnfahrplan. Hier noch die Broadcastgruppen über WhatsApp und Signal.\\
\vspace{1mm}

\begin{flushleft}
  \qrcode[height=20mm]{https://chat.whatsapp.com/KhDfCdUPRx17hmxWPSGhqO?mode=ems_qr_t} \href{https://chat.whatsapp.com/KhDfCdUPRx17hmxWPSGhqO}{<--------\textbf{Whatsapp Broadcast}}
  \begin{flushright}
    \vspace{-23.5mm}
    \href{https://signal.group/#CjQKIANRh68V0_YvlDQHd29ezVVIEeYPAfFAK6zVbxDcsLMcEhBsm9SxVk57Vsyo5QniX0GT}{\textbf{Signal Broadcast}-------->}\qrcode[height=20mm]{https://signal.group/#CjQKIANRh68V0_YvlDQHd29ezVVIEeYPAfFAK6zVbxDcsLMcEhBsm9SxVk57Vsyo5QniX0GT}  
  \end{flushright}
\end{flushleft}



\subsection{Studierwerkstatt}

\vspace{-3mm}
\textbf{Studierwerkstatt, deine Anlaufstelle zum Lernen}
\vspace{-2mm}

Die \textbf{Studierwerkstatt} findest du in \textbf{Raum C153}. Hier bekommst du Unterstützung in fast allen \textbf{Grundlagenfächern der Informatik}. Ob Nachhilfe, Lernberatung oder einfach gemeinsames Lernen, hier bist du richtig.
\vspace{-2mm}

\textbf{Öffnungszeiten}  
In der Vorlesungszeit ist die Studierwerkstatt von Montag bis Freitag von \textbf{13:00 bis 19:00 Uhr} geöffnet.  
Kurz vor der zweiten Prüfungsphase sind die Zeiten \textbf{10:00 bis 12:00 Uhr} und \textbf{13:00 bis 17:00 Uhr}.
\vspace{-2mm}

\textbf{Wer hilft dir?}  
Das Angebot wird ausschließlich von Studierenden betreut, die als Studentischen Hilfskräfte(SHKs) oder wissenschaftliche Hilfskräfte(WHKs) angestellt sind.
\vspace{-2mm}

\textbf{Wie findest du die richtige Person?}  
Der \textbf{Arbeitsplan} hängt direkt in der Studierwerkstatt an einem Whiteboard. Dort kannst du sehen, wann Tutorinnen und Tutoren für dein Fach da sind.
\vspace{-2mm}

\textbf{Kosten}  
Der Service ist für dich komplett kostenlos.
\vspace{-2mm}

\textbf{Kontakt}  
\vspace{-2mm}

Komm einfach vorbei. Für organisatorische Fragen kannst du dich per Mail an \\
\href{mailto:oliver.lanzerath@h-brs.de}{oliver.lanzerath@h-brs.de} (Oliver Lanzerath) oder  
\href{mailto:sigrid.weil@h-brs.de}{sigrid.weil@h-brs.de} (Sigrid Weil) wenden.

\subsection{Mensa \& Koff-in vom STWB}
\subsubsection{\href{https://www.studierendenwerk-bonn.de/essen-trinken/mensen-cafes/mensa-sankt-augustin}{Mensa vom STWB}}

\vspace{-3mm}
Direkt links vom Haupteingang findest du die Mensa. Dort gibt es von \textbf{Montag} bis \textbf{Freitag} Mittagessen.  
Das aktuelle Menü siehst du an den \textbf{Monitoren} im \textbf{Eingang der Mensa}, \textbf{online beim Studierendenwerk} oder direkt \textbf{vor unserem Büro} auf dem \textbf{Bildschirm} mit dem Wochenplan oder weiter unten im Text über den \textbf{QR-Code}.\\
Wenn du dein Essen ausgesucht hast, kannst du an die \textbf{SB-Kassen} oder die besetzten Kassen gehen. Bezahlen kannst du aktuell nur noch mit \textbf{Kreditkarte oder Girokarte} oder an den \textbf{SB-Kassen} mit deinem \textbf{Studierendenausweis}. Deinen Studierendenausweis kannst du an den Mensa-Automaten mit Hilfe deiner Bankkarte aufladen. Bargeld wird nicht mehr akzeptiert.

\vspace{-2mm}
Außerhalb von Öffnungszeiten der Mensa wird der Speisesaal als Lernraum angeboten und während der Prüfungsphasen als Prüfungsraum genutzt.

\textbf{Öffnungszeiten Mo–Fr}
\vspace{-2mm}

\begin{itemize}[noitemsep,nolistsep]
  \item Während des Semesters: \textbf{11:30–14:30 Uhr}
  \item Vorlesungsfreie Zeit: \textbf{11:30–14:00 Uhr}
  \item Prüfungsphase: \textbf{11:30–14:00 Uhr}
\end{itemize}
\vspace{-4pt}

\begin{center}
  \qrcode[height=30mm]{https://www.maxmanager.de/daten-extern/sw-bonn/pdf/wochenplaene/1/aktuell_de.pdf}\\
  \vspace{5mm}
  \textbf{ Dieser QRcode zeigt euch \underline{IMMER} den aktuellen Wocheplan der Mensa}\\
  \small\url{https://www.maxmanager.de/daten-extern/sw-bonn/pdf/wochenplaene/1/aktuell_de.pdf}\\
  \vspace{5mm}
  \textbf{ Dieser QRcode zeigt euch \underline{IMMER} den nächsten Wocheplan der Mensa}\\
  \small\url{https://www.maxmanager.de/daten-extern/sw-bonn/pdf/wochenplaene/1/naechste-woche_de.pdf}\\
  \vspace{1mm}
  \qrcode[height=30mm]{https://www.maxmanager.de/daten-extern/sw-bonn/pdf/wochenplaene/1/naechste-woche_de.pdf}\\
\end{center}
\vspace{-4pt}

\textbf{Tipp:} Zwischen \textbf{12:00 Uhr und 13:00 Uhr} gibt es den größten Andrang. Gehe entweder direkt nach den Vorlesungen oder so spät wie möglich, damit du nicht zu lange in der Schlange warten musst und die Stoßzeiten umgehst.

\subsubsection{Koff-in vom STWB}
Geöffnet \textbf{Mo–Fr}:
\begin{itemize}[noitemsep,nolistsep]
  \item Im Semester: \textbf{\textbf{08:00–16:00} Uhr}
  \item In den Semesterferien: \textbf{\textbf{08:00–14:00} Uhr}
  \item Prüfungsphasen: \textbf{\textbf{08:00–15:00} Uhr}
\end{itemize}

Wenn dir der \textbf{Andrang an der Mensa} zu viel wird oder du \textbf{zwischen deinen Vorlesungen} eine Stärkung brauchst kannst du auch zum \textbf{koff-in} gehen. Das ist wie ein kleines Café mit halbrunder Theke und gemütlichen Sitzmöglichkeiten. 
Den findest du entlang der Hochschulstraße weiter runter, unmittelbar neben der Mensa.

Dort gibt es belegte Brötchen, Sandwiches, Backwaren, Müsli-Joghurts, Süßwaren, kalte Getränke wie Kakao, Softdrinks oder Energy Drinks und auch Heißgetränke.

Bring gerne deinen eigenen Becher für Heißgetränke mit, sonst gibt es auch die \textbf{LogiCups Becher} die du für einen Aufpreis für 0,70€ bekommen kannst, wenn du mal nicht vor Ort sitzen möchtest. Bei der Rückgabe des Bechers bekommst du \textbf{0,50€} als \textbf{Pfandrückgabe} wieder.

Gezahlt kann nur noch digital per Bankkarte, Debitkarte oder mit deinem Studiausweis den du vorab an den Mensa-Automaten aufladen kannst. Dein Studiausweis, kann dich nicht nur ausweisen, sondern auch für dich bezahlen! (aber (ausschließlich) an \textbf{allen Orten vom Studierendenwerk Bonn} (\href{https://www.studierendenwerk-bonn.de/essen-trinken}{\textbf{Mensen, Cafés}}))

\begin{center}
  \qrcode[height=30mm]{https://www.studierendenwerk-bonn.de/essen-trinken}\\
  \href{https://www.studierendenwerk-bonn.de/essen-trinken}{\textbf{https://www.studierendenwerk-bonn.de/essen-trinken}}
\end{center}

\textbf{Tipp:} Ab und zu gibt es frische Waffeln, schnell sein lohnt sich!

Sollte selbst das Koff-in zu Stoßzeiten mal überfüllt sein oder ist die Schlange zu lang, gibt es für Heißgetränke sogar eine dritte Alternative die du in der Hochschule finden kannst. 

\paragraph{CaffKar vom Café ``das Cultura``}

Dieser Wagen wird von \textbf{8:00 Uhr bis 16:00 Uhr} von Studierenden und gelegentlich ``vom Besitzer`` geführt/bedient. Finden kannst du ihn draußen neben dem Haupteingang oder am Ende der Hochschulstraße. \\
Dort bekommst du in den \textbf{RECUP Bechern} ebenfalls diverse Heißgetränke, die aufwändiger und sogar mit Extras zubereitet werden, wie z.B. Sirups. Für die Becher dort gibt es ebenfalls ein \textbf{Pfand von 1€} den du bei Rückgabe wieder \textbf{vollständig zurückbekommst}.  



\newpage
\subsection{Bibliothek}

\subsubsection*{Bibliothek für Erstis}
\vspace{-4mm}
\begin{wrapfigure}{r}{0.45\textwidth}
  \vspace{-10pt}
  \includegraphics[width=\linewidth]{./2.0.0.0_content/photos/Bib/Ausleihe und Rückgabe.jpg}
\end{wrapfigure}

Willkommen an der H-BRS und in der Bibliothek!  

In Sankt Augustin findest du die Bibliothek im Hauptgebäude A im 1. OG und in Rheinbach im 1. OG des C-Gebäudes.  

Dein Studierendenausweis ist gleichzeitig auch dein Bibliotheksausweis. Du kannst beide Bibliotheken nutzen.  
Falls ein Buch nicht vor Ort ist, kannst du es über den \textbf{Fernleihservice} bestellen.  




\medskip
\textbf{Logins:}
\vspace{-3mm}
\begin{itemize}[itemsep=2pt,parsep=0pt,topsep=0pt]
  \item Bibliotheksservices (BibDiscover, BibPrint, BibCloud, Gruppenräume): Bibliotheksnummer + MIA-Passwort
  \item LEA: Fachbereichskürzel + MIA-Passwort
  \item Ausleihe am Selbstverbucher: Bibliothekskarte + Ausleih-PIN
  \item WLAN H-BRS: \url{https://info.h-brs.de/de/wlan/h-brs}
\end{itemize}

\vspace{-3mm}

\medskip
\textbf{Angebote:}
\vspace{-3mm}
\begin{itemize}[itemsep=2pt,parsep=0pt,topsep=0pt]%[noitemsep]
  \item Beratung zum wissenschaftlichen Arbeiten
  \item Software (Zotero, SciFlow)
  \item Datenbanken für die Recherche
  \item Lernplätze und Gruppenräume
  \item Ausleihe von Technik (Powerbanks, Kabel, Laptops …)
  \item Scannen, Drucken und Kopieren
  \item OBRS (One-Button-Recording-Studio)
\end{itemize}

\vspace{5mm}

\begin{center}
  \begin{minipage}{0.7\textwidth}
    \centering
    \textbf{Webseite der Bibliothek}\\
    \vspace{1mm}
    \qrcode[height=35mm]{https://www.h-brs.de/de/bibliothek}\\
    \vspace{1mm}
    \url{https://www.h-brs.de/de/bibliothek}
  \end{minipage}
\end{center}


\clearpage
%\subsubsection*{Gebühren und Kosten}
%\begin{wrapfigure}{l}{0.45\textwidth}
%  \vspace{-10pt}
%  \includegraphics[width=\linewidth]{./2.0.0.0_content/photos/Bib/whiteboard.jpg}
%\end{wrapfigure}

\begin{wrapstuff}[type=figure, l, width=0.5\textwidth, top=1]
    \setlength{\abovecaptionskip}{8pt plus 3pt minus 2pt}
    \centering
    \includegraphics[width=\textwidth]{./2.0.0.0_content/photos/Bib/Bibliothek Gruppe Lernen Imagekampagne 2023-05-24 Lichtenscheid 005.JPG}
    \\[0.5em]
    {\small \paragraph{Zu spät abgegeben?}
  \begin{itemize}[itemsep=2pt,parsep=0pt,topsep=0pt]
    \item bis 10 Tage: \textbf{2 €}
    \item bis 20 Tage: \textbf{5 €}
    \item bis 30 Tage: \textbf{10 €}
    \item mehr als 30 Tage: \textbf{20 €}
  \end{itemize}}
\end{wrapstuff}

Für alle Studis der H-BRS ist die Nutzung der Bib \textbf{komplett kostenlos}.  
Externe Nutzer:innen zahlen ein Jahresentgelt von \textbf{10 €}  
(Ausnahme: Studierende an staatlichen Hochschulen NRW).  

Du kannst Bücher und Medien jeweils nur für \textbf{einen Monat} ausleihen. Achte daher bei deinen Ausleihen darauf, dass du die geliehenen Sachen rechtzeitig zurückgibst, um Gebühren zu vermeiden.\\


%  \paragraph{Zu spät abgegeben?}
%  \begin{itemize}[itemsep=2pt,parsep=0pt,topsep=0pt]
%    \item bis 10 Tage: \textbf{2 €}
%    \item bis 20 Tage: \textbf{5 €}
%    \item bis 30 Tage: \textbf{10 €}
%    \item mehr als 30 Tage: \textbf{20 €}
%  \end{itemize}

\vspace{8mm}
  \paragraph{Weitere mögliche Kosten:}
  \vspace{-0,5mm}
  \begin{itemize}[itemsep=2pt,parsep=0pt,topsep=0pt]
    \item Fernleihe: \textbf{1,50 €}
    \item Neuer Ausweis: \textbf{10 €}
    \item Verlust/Ersatz: \textbf{25 €}
    \item Kopien/Ausdrucke: \textbf{0,05 € pro Seite}
    \item Extra-Service (bibliographische Auskünfte): \textbf{45 €/h} (min. 15 €)
  \end{itemize}

\vspace{3mm}
\begin{center}
  \qrcode[height=2cm]{https://www.h-brs.de/de/bib/gebuehren-und-kosten}\\
  \small\url{https://www.h-brs.de/de/bib/gebuehren-und-kosten}
\end{center}
\medskip
\textbf{Nützliche Links:}
\vspace{-3mm}
\begin{multicols}{2}
\begin{itemize}[itemsep=3pt,parsep=0pt,topsep=0pt]
  \item LEA: \url{https://lea.hochschule-bonn-rhein-sieg.de}
  \item Beratung: \url{https://www.h-brs.de/de/bib/beratungsangebot}
  \item Gruppenräume: \url{https://rrs.bib.h-brs.de/}
  \item Fernzugriff: \url{https://www.h-brs.de/de/bib/fernzugriff}
  \item BibPrint: \url{https://bib-print.bib.h-brs.de/user}
  \item BibCloud: \url{https://bib-cloud.bib.hochschule-bonn-rhein-sieg.de/login}
\end{itemize}
\end{multicols}


\subsubsection*{Regelmäßige Veranstaltungen}
\vspace{-5mm}
\begin{itemize}[itemsep=2pt,parsep=0pt,topsep=0pt]
  \item BibLounge – Schulungen zu Arbeiten, KI und Lernstrategien:  
    \url{https://www.h-brs.de/de/bib/biblounge-termine-und-themen}
  \item LiteraturKlatsch – literarisches Miteinander:  
    \url{https://www.h-brs.de/de/bib/literatur-klatsch-themen-und-termine}
  \item Zu Gast auf dem Sofa – Autorenlesungen:  
    \url{https://www.h-brs.de/de/bib/zu-gast-auf-dem-sofa}
\end{itemize}


\clearpage
\subsection{Sekretariat Informatik}

Das \textbf{Fachbereichssekretariat Informatik} in \textbf{C101} ist erster Ansprechpartner des Fachbereichs Informatik.  
Es unterstützt bei organisatorischen Fragen zu Studium, Lehrveranstaltungen, Lehraufträgen oder mitarbeiterrelevanten Themen und stellt den Kontakt zum Dekanat her. Außerdem ist hier die Abgabe und Entgegennahme von Post möglich (wenn nicht am Empfang über die Poststelle oder im Postkasten erhalten).


\textbf{Kontaktzeiten:}

\vspace{-3mm}
\begin{itemize}[noitemsep, nolistsep]
  \item Montag bis Mittwoch: 09:00 -- 12:00 Uhr
  \item Donnerstag: 09:00 -- 16:30 Uhr
  \item Vorlesungsfreie Zeit: Montag -- Freitag: 09:00 -- 12:00 Uhr
\end{itemize}

\textbf{E-Mail:} \href{mailto:fb02.sekretariat@h-brs.de}{fb02.sekretariat@h-brs.de}\\
\textbf{Webseite:} \href{https://www.h-brs.de/de/inf/sekretariat-des-fachbereichs-informatik}{Fachbereichssekretariat Informatik Webseite}

\vspace{-5mm}

\subsubsection*{Mitarbeiterinnen}

\begin{table}[!h]
\centering
\resizebox{\textwidth}{!}{%
\begin{tabular}{|l|l|l|l|l|}
\hline
\textbf{Name} & \textbf{Rolle} & \textbf{E-Mail} & \textbf{Raum} & \textbf{Telefon}  \\ \hline
Anne Maria Kaiser & Sekretariat & \href{mailto:anne.kaiser@h-brs.de}{\texttt{anne.kaiser@h-brs.de}} & C 101 & +49 2241 865 201 \\ \hline
Irina Malsam & Sekretariat & \href{mailto:irina.malsam@h-brs.de}{\texttt{irina.malsam@h-brs.de}} & C 101 & +49 2241 865 195 \\ \hline
Anette Schiffmann & Assistenz im Dekanat & \href{mailto:anette.schiffmann@h-brs.de}{\texttt{anette.schiffmann@h-brs.de}} & C 101 & +49 2241 865 201 \\ \hline
Kathrin Fetkenhauer & Assistenz der Prüfungsausschüsse & \href{mailto:kathrin.fetkenhauer@h-brs.de}{\texttt{kathrin.fetkenhauer@h-brs.de}} & C 105 & +49 2241 865 290 \\ \hline
\end{tabular}%
}
\end{table}



\vspace{-3mm}

\subsection{Empfang, Poststelle und Post}
\vspace{-1mm}

\subsubsection{Empfang}

\vspace{-4mm}
Der Empfang ist direkt geradeaus durch den Haupteingang in einem vitrinenähnlichen Häuschen.  
Dort kannst du:

\vspace{-4mm}
\begin{itemize}[itemsep=3pt, parsep=0pt, topsep=2pt]
  \item Schlüssel für Lernräume (über \textbf{SARBS} Online gebucht) abholen,
  \item deinen Studierendenausweis abholen,
  \item Fragen zu Räumen und Raumbuchungen stellen,
  \item Post abholen, über die Poststelle
  \item deine Fundsachen abholen (sofern sie dort abgegeben wurden).
\end{itemize}

\vspace{-2mm}
Der Empfang ist \textbf{bis 18 Uhr} besetzt. Danach übernimmt der Sicherheitsdienst: dann kannst du nur noch Schlüssel zurückgeben oder Briefe/Umschläge abgeben, die ggf. erst am nächsten Tag \textbf{gestempelt} werden.
\vspace{-5mm}

\subsubsection{Poststelle / Post}

\vspace{-2mm}
Die Poststelle ist von \textbf{7:30 Uhr} bis \textbf{15:30 Uhr} über den Empfang erreichbar. Dort kannst du von morgens bis \textbf{18 Uhr} Briefe, Pakete und andere Post abgeben. \textbf{(Sa-So geschlossen)}
Alternativ kannst du Briefe \textbf{links} neben dem Empfang einwerfen.  
Für Professor:innen gibt es Fächer direkt \textbf{vor dem Sekretariat} und \textbf{gegnüber an der Wand} im Flur.  
Am Empfang abgegebene Post \textbf{wird gestempelt}. Für \textbf{zeitkritische} Sendungen ist die Abgabe dort sinnvoller.  
Der Briefkasten vor dem Eingang wird \textbf{morgens} und \textbf{meist kurz vor Schließung} der Poststelle geleert.


\clearpage
\subsection{Servicepoint Fachbereich Informatik}
\textbf{\url{https://faq.infcs.de/servicepoint/}}

Wir sind die erste Anlaufstelle bei Problemen mit der \textbf{technischen Infrastruktur} des Fachbereichs.  

\medskip

\noindent
\textbf{Beispiele:}
\begin{itemize}[noitemsep,nolistsep]
  \item Ihr könnt euch in einen Account nicht mehr einloggen?
  \item Ein Upload-Portal funktioniert nicht wie erwartet?
  \item Ihr möchtet von zu Hause aus auf das Hochschul-Netzwerk zugreifen?
\end{itemize}

Außerdem unterstützen wir den \textbf{reibungslosen Ablauf des Studienbetriebs}.  

\medskip

\noindent
\textbf{Typische Fälle:}
\begin{itemize}[noitemsep,nolistsep]
  \item Ein Rechner in den Seminarräumen ist defekt.  
  \item Eure Lerngruppe möchte einen Raum leihen.  
  \item Dozent:innen kommen nicht mit der Raumtechnik klar.  
  \item Ihr müsst einen Vortrag halten, könnt euren Laptop aber nicht mit dem Beamer verbinden.  
\end{itemize}


In all diesen Situationen leisten wir \textbf{Soforthilfe}.\\
\vspace{5pt}
\textbf{Wofür können Studierende sich an uns wenden:}
\begin{itemize}[noitemsep,nolistsep]
  \item Einrichtung von \textbf{WLAN}, \textbf{VPN} und anderen Hochschuldiensten\\
  Wir unterstützen bei allen gängigen Geräten und Betriebssystemen, solange es das Studium betrifft!
  \item Probleme mit Hochschulaccounts:
  \begin{itemize}
    \item Welcher Account für welchen Dienst?
    \item Passwort vergessen?
  \end{itemize}
  \item Verleih von:
  \begin{itemize}
    \item Adaptern und Kabeln
    \item Seminarräumen für Lerngruppen
    \item Whiteboardmarkern und -reinigern
  \end{itemize}
\end{itemize}
Wie sind wir zu erreichen:
Werktags von \textbf{08:00 bis 18:00} in Raum \textbf{A102.1} (direkt über dem Haupteingang)
Oder per Mail an \href{mailto:servicepoint@mail.inf.h-brs.de}{\textbf{servicepoint@mail.inf.h-brs.de}}

\textbf{Tipps:} für einen guten Start ins Studium:
Macht euch direkt zum Start mit den Hochschulservices vertraut; nicht erst 5 Minuten vor der
ersten Abgabe. \text{;)}


\setlist[itemize]{label=$\square$}

\section{Check-Liste für das Studium}

\subsection{Organisatorisches}
\begin{itemize}
  \item \textbf{Immatrikulation abschließen} \\(Studierendenausweis abholen (wenn du diese nicht per Post erhalten hast), Uni-Account aktivieren (Das Passwort zur Aktivierung findest du in der Post bei, falls du nichts bekommst, bekommst du diesen später ausgehändigt. Zum Start am 1.Tag solltet du diesen Passwortzettel haben. Falls du diesen immer noch nicht erhalten hast, bitte beim FB02 Servicepoint abholen.))
  \item \textbf{Semesterbeitrag prüfen} \\ (falls noch offen/bezahlen, wie viel und wohin steht im ersten Brief, hebt den Brief auf!)
  \item \textbf{Studienbescheinigung speichern} \\ (wird z. B. für BAföG, Wohngeld, Krankenkasse, Studierendenwohnheim, Kindergeld, Halb-/Vollwaisenrente und viele weitere finanzielle Unterstützungen sowie Services wie Spotify oder Amazon benötigt)
  \item \textbf{Semesterticket aufladen/aktivieren} \\ (z. B. Deutschlandsemesterticket im Wallet des Smartphones oder als QR-Code, Anleitung im Heft)
  \item \textbf{Office/Windows Lizenzen sichern} \\ (Windows 10 Education und Office 365 mit 1TB OneDrive)
  \item \textbf{Andere Lizenzen mit Studierendenrabatt aktivieren/sichern} (JetBrains Suite, GitHub Pro, Fusion 360, Shapr3D etc.)
  \item \textbf{Hochschul-E-Mail regelmäßig nutzen} \\ (z. B. Weiterleitung einrichten oder IMAP über owa.stud.h-brs.de)
  \item \textbf{WLAN / Eduroam einrichten} \\ (z. B. Easyroam, H-BRS WLAN oder Bibliothek)
  \item \textbf{VPN einrichten} \\ (für Bibliothek, Fachbereich, interne Dienste, Lernmaterialien etc.)
\end{itemize}

\subsection{Studienorganisation}
\begin{itemize}
  \item \textbf{LEA (ILIAS) \& Apollo verstehen} \\ (Kursanmeldung, Materialien, Prüfungsanmeldung)
  \item \textbf{Modulhandbuch durchsehen} \\ (Prüfungsordnung + Studienverlaufsplan; am Ende dieses Heftes zu finden; bitte Quellen/Links/QR-Codes prüfen, da gedruckte Version evtl. veraltet)
  \item \textbf{Fristen kennen} \\ (Prüfungsanmeldung, Rücktrittsfristen(Apollo), Rückmeldung -> \href{https://www.h-brs.de/de/inf/fachbereichszeitplan-fuer-den-fachbereich-informatik}{FBR-Zeitplan Hier klicken!})
  \item \textbf{Erstsemesterveranstaltungen besuchen} \\ (Einführungsveranstaltungen, Vorkurse, Vorstellungen, Campusrallye, Ersti-Woche)
  \item \textbf{Stundenplan bauen} \\ (Vorlesungen, Übugen etc. eintragen) \\ Hier kannst du deinen Stundeplan extra für alle B.Sc und alle M.Sc erstellen mit \href{
https://github.com/sotterbeck/hbrs-cal-creator}{\textbf{[’’1. hbrs-cal-creator von sotterbeck’’]}} und \href{https://github.com/leumasme/hbrs-timetable}{\textbf{[’’2. hbrs-timetable von leumasme’’]}} -> Eintragen dann auf der Rückseite des Heftes
  \item \textbf{Bibliothekszugang einrichten} \\ (falls nicht schon geschehen)
\end{itemize}

\subsection{Bibliothek}
\begin{itemize}
  \item \textbf{Bibliothek (be)suchen}
  \item \textbf{PIN für die Ausleihe am Selbstverbucher erstellen} \\ (in der Bibliothek oder in BibDiscover \href{https://bib-discover.bib.h-brs.de}{bib-discover.bib.h-brs.de})
  \item \textbf{Login in BibDiscover und LEA testen}
  \item \textbf{In LEA nach Kursen suchen}
  \item \textbf{Diesen LEA-Kursen beitreten}
  \item \textbf{Bib-Fernzugriff einrichten}
\end{itemize}

\subsection{Fachschaft \& Campusleben}
\begin{itemize}
  \item \textbf{Fachschaft kennenlernen} (Raum, vorbeischauen, Social Media folgen)
  \item \textbf{Ersti-Heft und Angebote durchsehen} (z. B. Hilfe für das Studium, Kursliste und viele Antworten auf Fragen)
  \item \textbf{Ersti-Veranstaltungen besuchen} \\ (Ersti Begrüßung, Spieleabend(Chill \& Play), Filmabende(Unifilm))
  \item \textbf{E-Sports, GameDev, RedRocket Club, Motorsport, Chor \& weitere Gruppen anschauen} \\ (digital oder in Person, wenn sie vor Ort am Campus vertreten sind)
  \item \textbf{Wohnheim-Veranstaltungen Europaring mitnehmen} \\(Barabende, Filmabende, Spieleabende, Flohmarkt, Ausflüge etc.)
\end{itemize}

\clearpage
\subsection{Praktisches fürs Studium}
\begin{itemize}
  \item \textbf{Notizen-System einrichten} (z. B. Zettelkasten, OneNote, Obsidian, Notion)
  \item \textbf{Passwort-Manager nutzen} (z. B. Vaultwarden, Bitwarden, Proton Pass, KeyPass, um nicht überall dasselbe Passwort zu haben)
  \item \textbf{Software installieren} (IDE, LaTeX, Python/andere Programmiersprachen, GitHub-Zugriff)
  \item \textbf{VPN/Serverzugang testen} (Wireguard, OpenVPN, falls benötigt)
  \item \textbf{Uni-Drucker/Kopierer einrichten} (Zuhause, In der Bib oder Nachfragen ob man in der Fachschaft drucken darf)
  \item \textbf{Mensa-Plan checken} (vor der Mensa selbst oder am Info-Screen vor dem Büro der Fachschaft oder im Heft unter Mensa)
\end{itemize}

\vspace{-3mm}

\subsection{Finanzen \& Formalitäten}
\begin{itemize}
  \item \textbf{BAföG-Antrag stellen} (falls relevant)
  \item \textbf{Wohngeld beantragen} (falls du kein Anrecht auf BAföG hast, ist es einen Versuch wert, die meisten sind teils berechtigt)
  \item \textbf{Kindergeld verlängern} (Studienbescheinigung dem Amt zuschicken)
  \item \textbf{Krankenversicherung klären} \\(gesetzlich/privat, Studententarif, falls nicht familienversichert und regelmäßig Studienbescheinigung an die Krankenkasse senden)
  \item \textbf{Nebenjob / Werkstudent anmelden} (Steuerklasse, Freibeträge, Sozialversicherung beachten, falls unbedingt nötig neben der Uni zu arbeiten)
  \item \textbf{Bankdaten für Gehalt/Unterstützung prüfen} (falls du auf dich alleine gestellt bist, das betrifft mehr als du denkst)
\end{itemize}

\vspace{-3mm}


\subsection{Persönlich \& Sozial}
\begin{itemize}
  \item \textbf{Kontakte knüpfen} (Kommilitonen in Vorlesungen, Tutorien, Events, Fachschaft)
  \item \href{https://chat.whatsapp.com/KhDfCdUPRx17hmxWPSGhqO}{\textbf{WhatsApp-}}\href{https://signal.group/#CjQKIANRh68V0_YvlDQHd29ezVVIEeYPAfFAK6zVbxDcsLMcEhBsm9SxVk57Vsyo5QniX0GT}{\textbf{/Signal-}}\textbf{Gruppen beitreten} (Unsere Info Ticker, QRcodes hier im Heft unter wichtige Orte)
  \item \textbf{Events \& Partys besuchen} (z. B. Kneipentour, Campusfeste, Weihnachtsmarkt/Weihnachtsfeier,\\ Chill \& Play (unsere regelmäßigen Spieleabende))
  \item \textbf{Sportangebote / Hochschulsport ansehen} (Hier im Heft unter AStA)
  \item \textbf{Beratungsstellen merken}(psychologische Beratung, Schreibzentrum, Sprachzentrum etc.)
\end{itemize}

\setlist[itemize]{label=\textbullet}